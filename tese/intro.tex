\subsection{Context}   % English
 	Since the early days of human history, we tried to communicate over far locations, from smoke signals to letters delivered by messengers, real-time communications were limited or even nonexistent. Despite all the efforts made to improve communications, written communication could never replace face to face communication.

	With the advent of the telephone network, communications have taken a very important step for us to feel more connected with whom we communicate. Still, only the human voice was not enough, the invention of cameras and their consequent digitization were a huge step for real-time communications.

 	Although still nothing replace the physical contact with a person while we communicate, we are at a time where we can do more than just a visual and verbal communication.
 	
	In the past, handwritten documents were limited to a writer per page at a time. Writing a book collaboratively was a difficult task due to synchronism between writers.

	Today, we can achieve more, it is possible to write a document collaboratively, correct spelling mistakes without wasting paper, restructure text at any moment, add a video to a newspaper article and more. Although much of what was said seem banal nowadays, none of this was possible before the computer's invention. 

\subsection{Problem Statement} % English

	As communications technologies appeared, we adapted the way we communicate. This project aims to re-create how we communicate and make it a process more natural and easy to learn. 

	For multiple reasons, some people tend to forget what they ear or see, either due to health problems or lack of sleep, a real-time system is not appropriate for people with short term memory loss, an application that provides a way to remember our past communications would be a strong tool for not only to catch what we lost but also to enhance our knowledge.

	This project aims to extend audio, text and video communications in order to create more rich and collaborative interfaces, better content organization and time handling. All of these with help of only standard technologies like \ac{WebRTC}, any additional plug-in is avoidable, \emph{JavaScript} libraries will be preferred as they can be downloaded on the fly.  

\subsection{Thesis Contributions} % English

	A web application with an easy to learn user interface will be developed to accomplish our goals. Our application, unnamed yet, is supposed to run on most web browsers that are compatible with JavaScript, \ac{WebRTC}, \ac{HTML}5 and \ac{CSS}3.

	All the problems faced during the development and limitations will be reported on the essay so that a future project better then ours can be easily and better developed.

\subsection{Article Structure} % English

	The second section is dedicated to related work, which is divided into five subsections. The first subsection describes the problems that real-time communications face on nowadays internet, namely the \ac{IPv4} address exhaustion and the client server model constraints. The second subsection describes the \ac{WebRTC} technology and the protocols needed to implement our project. The third subsection addresses the signaling component which is not defined on \ac{WebRTC} specifications. The fourth subsection presents the evolution of multimedia content until the hypermedia, its capabilities, synchronization mechanisms and interactivity. The  fifth and last subsection explores streaming protocols for non-interactive multimedia and how to introduce the interactive component, another important aspects are the ability to control the time flux of a stream and collaborative application development.

	In the third section we propose an architecture in order to develop our Web Application including all the need infrastructure and software.

	The fourth section describes our work methodology and what is our plan in order to solve our problem.

	The fifth section presents the summary and conclusions of our work.

