\subsection{Context}   % English
 	Since the early days of Human History, we tried to communicate over far locations, from smoke signals to letters delivered by messengers, real-time communications were limited or even nonexistent. Despite all the efforts made to improve communications, written communication could never replace face to face communication.

	With the advent of the telephone network, communications have taken a very important step for us to feel more connected with whom we communicate. Still, only the human voice was not enough, the invention of cameras and consequent video digitization were a huge step for real-time communications.

	In the past, handwritten documents were limited to a writer per page at a time. Writing a book collaboratively was a difficult task due to synchronism between writers.

	Today, we can achieve more, it is possible to write a document collaboratively, correct spelling mistakes without wasting paper, restructure text at any moment, add a video to a newspaper article and more. Although much of what was said seem banal nowadays, none of this was possible before the computer's invention. 

	As Martin Geddes\cite{geddes} states, \say{No computer in our lifetimes will ever rival a human voice’s capacity to conveying rich and complex social and emotional meaning} , although nothing replace the physical contact with a person while we communicate, we are at a time when we can do more than just a visual and verbal communication, hypermedia can be added to video and voice in order to extend its value. The concept of structured voice and video synchronized with hypermedia is called hypervoice.

\subsection{Problem Statement} % English

	As communications technologies appeared, we adapted the way we communicate. This project don't aims to replace our video and audio communications, but to enrich them with hyper-media content and make them a more natural and easy to learn process. 

	For multiple reasons, some people tend to forget what they ear or see, either due to health problems or lack of sleep, a real-time system is not appropriate for people with short term memory loss, an application that provides a way to remember our past communications would be a strong tool for not only to catch what we lost but also to enhance our knowledge.

	This project aims to extend audio, text and video communications in order to create more rich and collaborative interfaces, better content organization and time handling. All of these with help of only standard technologies like \ac{WebRTC}, any additional plug-in is avoidable, \emph{JavaScript} libraries will be preferred as they can be downloaded on the fly.  

\subsection{Thesis Goals} % English

	A web application with an easy to learn user interface will be developed to accomplish solve our problem. Our application, unnamed yet, is supposed to run on most web browsers that are compatible with JavaScript, \ac{WebRTC}, \ac{HTML}5 and \ac{CSS}3.

	All the problems faced during the development and limitations will be reported on the essay so that a future project better then ours can be easily and better developed.

\subsection{Document Structure} % English

	Section \ref{related} is dedicated to related work, which is divided into five subsections. 

	Section \ref{early} describes the problems that real-time communications face on nowadays internet, namely the \ac{IPv4} address exhaustion and the client server model constraints. 

	Section \ref{rtc} describes the \ac{WebRTC} technology and the protocols needed to implement our project. 

	Section \ref{signaling} addresses the signaling component of chat applications, which is not defined on \ac{WebRTC} specifications. 

	Section \ref{hypermedia} presents the evolution of multimedia content until the hypermedia, its capabilities, synchronization mechanisms and interactivity. 

	Section \ref{collab} explores streaming protocols for non-interactive multimedia and how to introduce the interactive component, another important aspects are the ability to control the time flux of a stream and collaborative application development.

	In section \ref{arch} we propose an architecture in order to develop our Web Application including all the need infrastructure and software.

	Section \ref{meth} describes our work methodology and what is our plan in order to solve our problem.

	Section \ref{concl} presents the summary and conclusions of our work.

