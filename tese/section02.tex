\subsection{Signaling, meet and get to know}


  Signaling is the most important process for applications to exchange connection information about peers and servers, their capabilities and meta-data.

  \ac{WebRTC} doesn't implement signaling, different applications may require different protocols, there is no single answer that fits all problems. Amongst multiple options, signaling can be done by using \ac{SIP}, \ac{XMPP}, \textit{WebSockets}, \textit{Socket.io} or by implementing a custom protocol.

  \ac{WebRTC} uses \ac{SDP} \cite{rfc4566} to define peer connection properties such as types of supported media, codecs, protocols used and network information. An \ac{SDP} offer describes to other peers the expected type of communication and its details, such as used transport protocols, codecs, security and other.

  One of signaling requisites is bi-directional communication over \ac{HTTP}. \ac{HTTP} works on a request followed by a server response, by other words, follows a unidirectional communication. Sometimes it's required that some informations are obtained in real time, as we saw, some \ac{NAT}'s don't suport callbacks from servers, one technique to overcome this problem is polling.

  Polling consists on sending periodic messages that the server responds immediately with empty content or fresh information. Real time communications are unpredictable, if the time between periodic requests is short, most of the time the server will return empty results wasting network bandwidth and energy. On the other hand, if the time between periodic requests is large, newer messages may arrive later.

  A technique called long polling consists on making the server hold the request until there is fresh information or expiring after some time, after the message receipt, the client makes another request. Long polling technique results on a better network usage and a faster server response, but both simple polling and long polling requests are sent with \ac{HTTP} headers, which adds data overhead, especially for short messages.

  The WebSocket protocol \cite{rfc6455} provides bidirectional communications over a full-duplex socket channel. WebSocket handshake phase specifies an \ac{HTTP} header in order to upgrade to \textit{WebSocket} type of communication, the remainder messages are done without \ac{HTTP} headers, which leads to much smaller messages and better network usage. WebSockets may not be available on every web browser, frameworks like \textit{socket.io}\footnote{\url{http://socket.io/}} and \textit{SockJS}\footnote{\url{http://github.com/sockjs}} uses \ac{HTTP} when there is no support for WebSockets. 

  \ac{BOSH}\cite{xep0124} is a technique based on long polling that uses two socket connections and allow sending client messages to server while a previous request is held.

  \ac{BOSH} specification assumes that a connection manager is implemented to handle \ac{HTTP} connections. This connection manager is basically a translator from \ac{HTTP} to raw message so the server may think this communication is performed over \ac{TCP}.

  When the connection manager holds for a response for too long it responds with an empty body, this technique prevents an \ac{HTTP} session from expiring when the client is waiting for a response, thus expanding the session time. Expiring sessions can be expensive due to the overhead of establishing new connections, it worsens even more when \ac{HTTP} is used over \ac{SSL}.

  If the server is holding a request, it maintains a second connection to receive more requests from the same client. The request on hold returns immediately with a possible empty body leaving its socket free, while the second connection serves the polling loop. The exchange of roles of those two connections allow to pull data from multiple contexts instead of being locked in just one.

  If the client has data to send while a request is still open, it establishes a second socket connection to the connection manager to send a new request. The connection manager immediately responds to the previously held request (possibly with no data) and holds open this new request. This results in the connections switching roles.

  \ac{SIP} \cite{rfc3261} is protocol used for negotiation, creation, modification and finalization of communication sessions between users. \ac{SIP} follows a client/server architecture with \ac{HTTP} like messages and it can be used as signaling protocol. The advantage of \ac{SIP} is the ability to make video and voice call's applications over \ac{IP} networks.

  The working group \ac{SIMPLE}\footnote{\url{https://datatracker.ietf.org/wg/simple/documents/}} proposed the creation of \ac{SIP} extensions, namely presence information \cite{rfc5263} and instant messaging \cite{rfc3428}.

  \ac{SIP} is used in \ac{VoIP} applications due to its compatibility with \ac{PSTN}. Service providers making their \ac{SIP} infrastructures available through WebSockets. Frameworks like \textit{jsSIP}\footnote{\url{http://jssip.net/}}, \textit{QoffeeSIP}\footnote{\url{http://qoffeesip.quobis.com/}} and \textit{sipML5}\footnote{\url{http://sipml5.org/}} are used on client side to parse and encode \ac{SIP} messages, making \ac{SIP} accessible to web based applications. 

  \ac{SIP} with \textit{WebSockets} can be used as a signaling method for \ac{WebRTC} applications, it allows web browsers to have audio, video and \ac{SMS} capabilities like mobile phones. For instance, it's possible to inter-operate web communications with \ac{SIP} networks, mobile and fixed phones.

  \ac{XMPP} was initially developed for instant messaging and presence (Jabber\footnote{\url{http://jabber.org/}}). It is nowadays an open technology for standardized, decentralized, secure and extensible real-time communications. 

  \ac{XMPP} messages are \ac{XML} based, which are attractive for applications that need structured messages and rich hypermedia. \ac{XMPP} advantage is the addition of extensions, for example \cite{xep0096}, which adds file transfer capabilities between two entities and \cite{xep0045} which enables multi-user chat.

  \ac{XMPP}'s bi-directional communication is achieved through \ac{BOSH} \cite{xep0206}, which basically consists on long polling. This kind of communication is also possible through WebSockets \cite{rfc7395}.

  Amongst multiple XMPP server solutions are: \textit{ejabberd}\footnote{\url{http://jabberd.im/}}, \textit{Metronome}\footnote{\url{http://lightwitch.org/metronome}}, \textit{Openfire}\footnote{\url{http://igniterealtime.org/projects/openfire/}} and \textit{Prosody}\footnote{\url{http://prosody.im/}}. \textit{Ejabberd} is the one that implements more \ac{RFC} specifications and \ac{XEP}s\footnote{\url{http://en.wikipedia.org/wiki/Comparison_of_XMPP_server_software}}.

  A more interesting approach for signaling would be \ac{SigOfly}\cite{sigofly} which allows inter-domain real-time communications. \ac{SigOfly} achieves inter-domain communication by making use of the \textit{Identity Providers} of each peer. 

  The caller entity downloads a page with all the code need, also known as messaging stub, to communicate with the called party, this code contains an implementation of the signaling protocol used in order to communicate to the called peer. If the called party domain is being overused it is possible to switch the caller and called parties role, after that the called entity downloads the stub code from the caller domain instead.

  \ac{SigOfly} is an approach very flexible because participants on a video call are not tied to just one type o signaling implementation. Another important aspect of  \ac{SigOfly} is the ability to perform multi-party conversations either through a \textit{Mesh Topology} or a \textit{Multipoint Control Unit}.


