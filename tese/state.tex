\documentclass[11pt,twocolumn]{article}




\usepackage[utf8x]{inputenc}
\usepackage[T1]{fontenc}
\usepackage[fleqn]{amsmath}
\usepackage{amssymb}
\usepackage{listings}
\usepackage{color}
\usepackage{makeidx}
\usepackage{blindtext}
\usepackage{float}
\usepackage{graphicx}
\usepackage{mathtools}
\usepackage{caption}
\usepackage{subcaption}
\usepackage[hmargin=3cm,vmargin=3.5cm]{geometry}
\usepackage{indentfirst}
\definecolor{dkgreen}{rgb}{0,0.6,0}
\definecolor{gray}{rgb}{0.5,0.5,0.5}
\definecolor{mauve}{rgb}{0.58,0,0.82}
\usepackage{eurosym}
\usepackage{url}





\lstset{frame=tb,
  language=C,
  aboveskip=3mm,
  belowskip=3mm,
  showstringspaces=false,
  columns=flexible,
  basicstyle={\small\ttfamily},
  numbers=none,
  numberstyle=\tiny\color{gray},
  keywordstyle=\color{blue},
  commentstyle=\color{dkgreen},
  stringstyle=\color{mauve},
  breaklines=true,
  breakatwhitespace=true
  tabsize=3
}

\usepackage{titlesec}
\titleformat{\section}{\hrule \large\bfseries}{\thesection}{1em}{}

\title{Conversas Hiperligadas: Novo Paradigma de Comunicação e Colaboração, potenciado pela Tecnologia WebRTC}
\author{
	Henrique Lopes Rocha \\ hdlopesrocha91@gmail.com
}




\date{\today}
\begin{document}

\begingroup
\let\onecolumn\twocolumn

	\maketitle


\begin{abstract}
Lorem ipsum dolor sit amet, consectetuer adipiscing elit. Etiam lobortis facilisis sem. Nullam nec mi et neque pharetra sollicitudin. Praesent imperdiet mi nec ante. Donec ullamcorper, felis non sodales commodo, lectus velit ultrices augue, a dignissim nibh lectus placerat pede.
\end{abstract}

\endgroup


\section{Introduction}
The need to build a global comunications network in an era when almost nobody had access to it, caused that some protocols weren't suitable for a huge increase on the amount of publicly known users. IPv4 limits the number of public addresses in such a way that today is scarse \cite{ipv4}. One way to overcome this problem was the development of a mechanism that groups multiple address into a single one, the machine that is assigned that address is then responsible to redirect messages to members of its group through their private addresses, each member of the private network is identified publicly by the same IP address but different port, this technique is also know as Network Address Translation (NAT).

Initially NAT offered an alternative for address exhaustion and a false sensation of better security, asymmetric NAT became a vulgar configuration on the web. As a direct result, problems started to appear, the amount of ports that IP disponibilizes is also small compared to our current needs, worse than that, NAT also difficult end-to-end communication, forcing most of applications that follows this model to be implemented ineffectively.

Applications based on multimedia and file sharing were one of the most strained by NAT. Those kind applications require real time communication in order to achieve the best performance. STUN and TURN \cite{natvoip} servers are a possible solution to overpass NAT, although, none of those can establish direct connections on multiple level NATs.

Most of client-server applications aren't affected by NAT when the servers are public, but they're inadequate for real time communication between two private endpoints. Clearly this type of communication requires a more expensive infrastructure and, in most cases, more network usage, leading to a worse quality of service. The requirements of video communication makes this kind of model unsuitable.

On the other hand, TURN uses public servers to redirect traffic between private endpoints, it may use a P2P network relay to find the best peer but, after that, the behavior is much like client-server. Direct communication is only achieved by STUN when NAT is a type \textit{full cone}. {\color{red} [reference to NAT types]}. ICE uses STUN when it's possible and TURN otherwise.

When connection is established, either in a direct or indirect way (via TURN servers), WebRTC cames to simplify how audio and video are transmited. 

\section{Hypermedia}

  Since the early days of video technology, one of the problems that raised with it consisted of how to add more information onto it without generating multiple versions. Some implementations like \cite{embedded} added hypermedia information to empty space present on MPEG frames, the MPEG coder and decoder were changed in order to handle hypermedia content.

  The need to translate movies, raised the problem whether it is apropriate to change the original video or audio. For example subtitles should be an entity independent from the video, in order to be personalized or replaced easily.
 
  Amongst multiple formats for subtitles, SAMI, SRT and SUB are used by video players that support them. Although those formats have styling available, they are quite limited to text. 

  HyVAL\cite{hyval} was proposed for modeling composition and interaction of video, this kind of video is also called hypervideo, the structure of HyVAL is derived from traditional video, which divides video into segments, scenes, shots and frames hierarchically. This approach is quite limitative if we want to apply hypervideo concepts to videos that don't follow this structure.

  SMIL was introduced to describe temporal behavior of multimedia, for instance, it could be used to overlay subtitles on films. With SMIL it's possible to synchronize multiple sections of video, either in parallel or in sequence, reproduce a different audio, overlay user interface elements with hyperlinks, amongst multiple other functionalities.

  In order to create a multimedia rich hypercall, SMIL fits our goals, but it lacks on browser compatibility. Ambulant\cite{ambulant} was one of the SMIL players that were developed for browsers, although this player implements most of SMIL 3.0 specifications, it needs to be installed on browsers as a plugin.

  SmillingWeb attempts to implement SMIL 3.0 with javascript and jQuery, which doesn't need to be installed and shouldn't have incompatibility issues. But SmillingWeb isn't fully implemented yet and their scheduler engine loads the SMIL file only once, which could raise problems when leading with SMIL changes in real time.  

  SVG is a format that implements part of SMIL, it allows dynamic changes to inner content in real time, besides that, it also allows to call javascript functions on events such as animation end, mouse over and mouse click.

  Video functionalities are already embedded in HTML5, like SVG it is also possible to bind javascript functions for different kinds of events over videos.

  By using solid technologies like Javascript, CSS, SVG and HTML5 in a coordinated way, it's possible to raise communications to a new level. For example, with API's like WebGL, it is now possible to manipulate a three dimensional environment in the context of a hypercall. Another example would be a collaborative spreadsheet using WebRTC. With this, hypercalls are not limited to only video, audio and text.

  In this project our goal is to enrich hypercalls with no limits, every user should be free to choose how it wants to be contacted.

  In order to give users a personalized communication channel, each user must have a personal web page where its available plugins could be downloaded from other peers, after that they can talk in the same language whatever it is.




\begin{thebibliography}{9}


\bibitem{ipv4} Next Generation Internet: IPv4 Address Exhaustion, Mitigation Strategies and Implications for the U.S. - An IEEE-USA White Paper - 2009
\bibitem{natvoip} How NAT-Compatible Are VoIP Applications? Ying-Dar Lin, Chien-Chao Tseng, Cheng-Yuan Ho, and Yu-Hsien Wu, National Chiao Tung University
\bibitem{hyval} A Structured Document Model for Authoring Video-based Hypermedia
\bibitem{embedded} Combinando TV Interativa e Hipervídeo - André Leon S. Gradvohl & Yuzo Iano, Member, IEEE
\bibitem{ambulant} A SMIL player for any web browser - Ombretta Gaggi and Luca Danese


\end{thebibliography}

\end{document}
