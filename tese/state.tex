\documentclass[11pt,twocolumn]{article}




\usepackage[utf8x]{inputenc}
\usepackage[T1]{fontenc}
\usepackage[fleqn]{amsmath}
\usepackage{amssymb}
\usepackage{listings}
\usepackage{color}
\usepackage{makeidx}
\usepackage{blindtext}
\usepackage{float}
\usepackage{graphicx}
\usepackage{mathtools}
\usepackage{caption}
\usepackage{subcaption}
\usepackage[hmargin=3cm,vmargin=3.5cm]{geometry}
\usepackage{indentfirst}
\definecolor{dkgreen}{rgb}{0,0.6,0}
\definecolor{gray}{rgb}{0.5,0.5,0.5}
\definecolor{mauve}{rgb}{0.58,0,0.82}
\usepackage{eurosym}
\usepackage{url}





\lstset{frame=tb,
  language=C,
  aboveskip=3mm,
  belowskip=3mm,
  showstringspaces=false,
  columns=flexible,
  basicstyle={\small\ttfamily},
  numbers=none,
  numberstyle=\tiny\color{gray},
  keywordstyle=\color{blue},
  commentstyle=\color{dkgreen},
  stringstyle=\color{mauve},
  breaklines=true,
  breakatwhitespace=true
  tabsize=3
}

\usepackage{titlesec}
\titleformat{\section}{\hrule \large\bfseries}{\thesection}{1em}{}

\title{Conversas Hiperligadas: Novo Paradigma de Comunicação e Colaboração, potenciado pela Tecnologia WebRTC}
\author{
	Henrique Lopes Rocha \\ hdlopesrocha91@gmail.com
}




\date{\today}
\begin{document}

\begingroup
\let\onecolumn\twocolumn

	\maketitle


\begin{abstract}
Lorem ipsum dolor sit amet, consectetuer adipiscing elit. Etiam lobortis facilisis sem. Nullam nec mi et neque pharetra sollicitudin. Praesent imperdiet mi nec ante. Donec ullamcorper, felis non sodales commodo, lectus velit ultrices augue, a dignissim nibh lectus placerat pede.
\end{abstract}

\endgroup


\section{Introduction}
The need to build a global comunication network in an era when almost nobody had access to it, caused that some protocols weren't suitable for a huge increase on the amount of publicly known users. IPv4 limits the number of public addresses in such a way that today are scarse \cite{ipv4}. One way to overcome this problem was the development of a mechanism that groups multiple address into a single one, the machine that is assigned that address is then responsible to redirect messages to members of its group through their private addresses, each element is identified publicly by the same IP address but different ports, this technique is also know as Network Address Translation (NAT).

Initially NAT offered an alternative for address exhaustion and a false sensation of better security, asymetric NAT became a vulgar configuration on the web. As a direct result, problems started to appear, the amount of ports that IP disponibilizes is also small compared to our current needs, worse than that, NAT also difficults end-to-end communication, forcing most of applications that follows this model to be implemented unificiently.

Applications based on multimedia and file sharing were one of the most strained by NAT. Those kind applications requires real time communication in order to achieve the best performance. STUN and TURN \cite{natvoip} servers are a possible solution to overpass NAT, although, none of those an establish direct connections on multiple level NATs.

Most of client-server applications aren't affected by NAT when the servers are public, but they aren't suitable for real time comunication between two private end points. Clearly this type of communication requires a more expensive infrastructure and, at most cases, more network usage, leading to a worse quality of service. The requirements of video comunication makes this kind of model out of question.

On the other hand, TURN uses public servers to redirect traffic between private end points, it may use a P2P network relay to find the best peer, but after that the behaviour is much like client-server. Direct communication is only achieved by STUN when NAT is type \textit{full cone}. {\color{red} [reference to NAT types]}. ICE uses STUN when it's possible and TURN otherwise.

When connection is established, either direct or indirect (via TURN servers), WebRTC cames to simplify how audio and video are transmited. 

\section{Hypemedia}

Although video and audio is almost standardized, WebRTC allows us to enrich web communications by overlaying multiple elements such as buttons, text and more complex user interfaces.

In the past, multiple formats were proposed to deal with hypermedia.

HyVAL\cite{hyval} was proposed for modeling composition of video. The definition of a format must be done with carefoul. When a format estabilished for a document, it is expected that a parser knows how to convert text into behaviour, formats like XML are quite limitative, they cannot reproduce actions that were not previously implemented.

Our solution to overcome format limitation is to take advantage of languages like javascript, like any webpage, user interface elements are defined by html tags, more complex actions like animations can be easily implemented via CSS and javascript, without ever defining a new standard for structured hypermedia.

By using this already solidified web technologies, it's possible to raise communications to a new level. For example, with API's like WebGL, it is now possible to manipulate a three dimensional environment in the context of an hypercall. Another example would be a colaborative spreadsheet using WebRTC. With this, hypercalls are not limited to only video, audio and text.


In this project our goal is to enrich hypercalls with no limits, every user should be free to choose how it whant to be contacted. 

Consider this case study, say that Alice whants to call Bob. Alice starts a WebRTC connection with Bob, at this point Alice doesn't know how the communication will proceed. Bob created a personal web page where he defined how he whants to be contacted. Alice downloads Bob's personal webpage and finds what kind of interactions could be done, for instance she knows that she can contact Bob via video streaming or can browse past videos. While Bob is connected, he can add links on videos, organise its content and enable different plugins for communication.

\begin{thebibliography}{9}


\bibitem{ipv4} Next Generation Internet: IPv4 Address Exhaustion, Mitigation Strategies and Implications for the U.S. - An IEEE-USA White Paper - 2009
\bibitem{natvoip} How NAT-Compatible Are VoIP Applications? Ying-Dar Lin, Chien-Chao Tseng, Cheng-Yuan Ho, and Yu-Hsien Wu, National Chiao Tung University
\bibitem{hyval} A Structured Document Model for Authoring Video-based Hypermedia

\end{thebibliography}

\end{document}
