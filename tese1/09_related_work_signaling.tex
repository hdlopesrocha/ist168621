\subsection{Signaling: meet and get to know}
\label{signaling}


  Signaling is the process by which applications exchange connection information about peers and servers, their capabilities and meta-data.
  \ac{WebRTC} doesn't implement signaling, as different applications may require different protocols, and there is no single answer that fits all problems.
  Multiple options are available for signaling, which can be performed using \ac{SIP}, \ac{XMPP}, \emph{WebSockets}, \emph{Socket.io} or by implementing a custom protocol.

  \ac{WebRTC} uses \ac{SDP} \cite{rfc4566} to define peer connection properties such as types of supported media, codecs, protocols used and network information. An \ac{SDP} offer describes to other peers the expected type of communication and its details, such as used transport protocols, codecs, security and other.

  One of \ac{WebRTC} signaling's requisites is bi-directional communication over \ac{HTTP}.
%RP acrescentei o webrtc pois isto não é um requisito geral de signaling. era o que querias dizer?
%HR sim é isso
  \ac{HTTP} uses a request-response paradigm, where a request is sent by the client, followed by a server response.
  %in other words, it follows an unidirectional communication.
%RP de certeza que queres dizer que o http é unidirecional? 
%HR era mais o facto de o servidor não poder comunicar com o cliente sem que o cliente comunique primeiro (tem que haver uma iniciativa por parte do cliente)
  Sometimes it is required that some information be obtained in real time, but we saw, some \ac{NAT}'s do not support callbacks from servers, preventing them from notifying clients as soon as an event occures.
One technique to overcome this problem is polling.
  %RP num documento formal não se usam contrações como it's.
  %RP o teu ingles é parco em artigos. Faltam muitos the, of, it, etc.
  
Polling consists on sending periodic messages to which the server responds immediately with empty content or fresh information. Text and presence messages are unpredictable, if the time between periodic requests is short, most of the time the server will return empty results wasting network bandwidth and energy. On the other hand, if the time between periodic requests is large, newer messages may arrive too late.
%RP  Real time communications are unpredictable. Normalmente falamos de audio e video, que são previsíveis. Estás a referir-te às comunicações em si ou ao signaling e estabelecimento?

A technique called long polling consists on making the server hold the request until there is fresh information or expiring it after some time.
As soon as it receives the reply, the client makes another request. Long polling technique results on a better network usage and a faster server response, but both simple polling and long polling requests are sent with \ac{HTTP} headers, which add data overhead, especially for short messages.

The WebSocket protocol allows bidirectional communications over a full-duplex socket channel \cite{rfc6455}.
%RP coloca as citações no final da frase. Pode haver excepções onde a citação apenas cobre parte da frase ou precisas de colocar várias a dizer coisas diferentes.
WebSocket handshake phase specifies a \ac{HTTP} header in order to upgrade to \emph{WebSocket} type of communication, but the remainder messages are exchanged without \ac{HTTP} headers, which leads to much smaller messages and better network usage. WebSockets may not be available on every web browser, frameworks like \emph{socket.io}\footnote{\url{http://socket.io/}(accessed June 1, 2015).} and \emph{SockJS}\footnote{\url{http://github.com/sockjs}(accessed June 1, 2015).} fall back to using \ac{HTTP} when there is no support for WebSockets. 

  \ac{BOSH}\cite{xep0124} is a technique based on long polling that uses two socket connections and allows sending client messages to the server while a previous request is held.
  The \ac{BOSH} specification assumes that a connection manager is implemented to handle \ac{HTTP} connections. This connection manager is basically a translator from \ac{HTTP} to raw message so that the server may be implemented as if this communication is performed over \ac{TCP}.
  When the connection manager holds for a response for too long, it responds with an empty body, this technique prevents an \ac{HTTP} session from expiring when the client is waiting for a response, thus expanding the session time. Expiring sessions can be expensive due to the overhead of establishing new connections, which is even worse when \ac{HTTP} is used over \ac{SSL}.

  If the server is holding a request, it maintains a second connection to receive more requests from the same client. The request on hold returns immediately with a possible empty body leaving its socket free, while the second connection serves the polling loop. The exchange of roles of those two connections allow to pull data from multiple contexts instead of being locked in just one.
  %RP estas duas frases dizem o mesmo da perspectiva do cliente e do servidor. Não consegues simplificar numa só?

  \ac{SIP} \cite{rfc3261} is a protocol used for negotiation, creation, modification and finalization of communication sessions between users. \ac{SIP} follows a client/server architecture with \ac{HTTP} like messages and it can be used as a signaling protocol. The advantage of \ac{SIP} is the ability to make video and voice call's applications over \ac{IP} networks.

  The working group \ac{SIMPLE}\footnote{\url{https://datatracker.ietf.org/wg/simple/documents/}(accessed June 1, 2015).} proposed the creation of \ac{SIP} extensions, namely presence information \cite{rfc5263} and instant messaging \cite{rfc3428}.

  \ac{SIP} is used in \ac{VoIP} applications due to its compatibility with the \ac{PSTN}.
  Service providers are making their \ac{SIP} infrastructures available through WebSockets.
%RP a frase anterior é estranha
  Frameworks like \emph{jsSIP}\footnote{\url{http://jssip.net/}(accessed June 1, 2015).}, \emph{QoffeeSIP}\footnote{\url{http://qoffeesip.quobis.com/}(accessed June 1, 2015).} and \emph{sipML5}\footnote{\url{http://sipml5.org/}(accessed June 1, 2015).} are used on the client side to parse and encode \ac{SIP} messages, making \ac{SIP} accessible to web based applications. 
%RP quando usas footnotes para indicar sites deves acrescentar ``last accessed (data)''. Isto é importante pois os sites são documentos efémeros que podem dizer uma coisa diferente mais tarde.
  \ac{SIP} with \emph{WebSockets} can be used as a signaling method for \ac{WebRTC} applications, it allows web browsers to have audio, video and \ac{SMS} capabilities like mobile phones. For instance, it's possible to inter-operate web communications with \ac{SIP} networks, mobile and fixed phones.

  \ac{XMPP} was initially developed for instant messaging and presence (Jabber\footnote{\url{http://jabber.org/}(accessed June 1, 2015).}). It is nowadays an open technology for standardized, decentralized, secure and extensible real-time communications. 
  \ac{XMPP} messages are \ac{XML} based, which is attractive for applications that need structured messages and rich hypermedia. Another advantage of \ac{XMPP} is the addition of extensions, for example \cite{xep0096}, which adds file transfer capabilities between two entities and \cite{xep0045} which enables multi-user chat.
  \ac{XMPP}'s bi-directional communication over \ac{HTTP} is achieved through \ac{BOSH} \cite{xep0206}.
  %RP falta um `` bi-directional communication OVER HTTP is achieved through''?
  %RP BOSH já foi explicado antes pelo que não vale a pena voltar a falar em long-pooling
  This kind of communication is also possible through WebSockets \cite{rfc7395}.
  Today, multiple XMPP server implementations exists, such as: \emph{ejabberd}\footnote{\url{http://jabberd.im/}(accessed June 1, 2015).}, \emph{Metronome}\footnote{\url{http://lightwitch.org/metronome}(accessed June 1, 2015).}, \emph{Openfire}\footnote{\url{http://igniterealtime.org/projects/openfire/}(accessed June 1, 2015).} and \emph{Prosody}\footnote{\url{http://prosody.im/}(accessed June 1, 2015).}. \emph{Ejabberd} is the server that implements more \ac{RFC} specifications and \ac{XEP}s\footnote{\url{http://en.wikipedia.org/wiki/Comparison_of_XMPP_server_software}(accessed June 1, 2015).}.
  %RP fazes muitos paragrafos. Um paragrafo por tecnologia é normalmente suficiente se não forem muito grandes.
  
  Another interesting approach for signaling would be \ac{SigOfly} which allows inter-domain real-time communications, while abstracting the protocol used~\cite{sigofly}.
%RP o ~ antes do cite faz com que não haja um espaço mas não haja uma mudança de linha entre a última palavra e a citação
  \ac{SigOfly} provides inter-domain communication by making use of the \emph{Identity Providers} of each peer. 
  The caller entity downloads a page with all the code needed, also known as messaging stub, to communicate with the called party.
  This code contains an implementation of the signaling protocol used in order to communicate to the called peer. If the called party domain is being overused, it is possible to switch the caller and called parties role, after that the called entity downloads the stub code from the caller domain instead.
  \ac{SigOfly} is an approach very flexible because participants on a video call are not tied to just one type of signaling implementation. Another important aspect of \ac{SigOfly} is the ability to perform multi-party conversations either through a \emph{Mesh Topology} or a \emph{Multipoint Control Unit}.
%RP os outros não permitem multi-party?
%HR adicionei ao Hangouts e Skype, quanto ao XMPP já digo que existe uma extensão que permite multi-user chat