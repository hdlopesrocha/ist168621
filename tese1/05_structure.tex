
\subsection{Document Structure} % English
This document is structured as follows. Section \ref{related} presents the related work.
%	Section \ref{early} describes the problems that real-time communications face on nowadays internet, namely the \ac{IPv4} address exhaustion and the client server model constraints. 
%	Section \ref{rtc} describes the \ac{WebRTC} technology and the protocols needed to implement our project. 
%	Section \ref{signaling} addresses the signaling component of chat applications, which is not defined on \ac{WebRTC} specifications. 
%	Section \ref{hypermedia} presents the evolution of multimedia content until the hypermedia, its capabilities, synchronization mechanisms and interactivity. 
%	Section \ref{collab} explores streaming protocols for non-interactive multimedia and how to introduce the interactive component, another important aspects are the ability to control the time flux of a stream and collaborative application development.
%RP este detalhe sobre a estrutura da sec. do estado da arte vai para o início dessa secção.   
	In Section \ref{arch}, we propose an architecture for an Web Application that fulfills the goals of this thesis, including all the need infrastructure and software.
	Section \ref{meth} describes our work methodology and our plan for implementing and validating our proposed architecture.
	Section \ref{concl} presents the summary and conclusions of our work.

