---------------------------------------------------------------

\section{Backend}


\subsection{Our approach}

In order to implement a solution without the \textit{XMPP}'s limitations that we presented on the previous section, we decided to implement our own backend. 

One important aspect that we should take into account is the scalability of our solution. All the information that we need to store relative to the users and chat rooms should be stored on a scalable database. 

As our database grows, we may need to distribue the load among multiple servers, which can be achieved on SQL servers by spliting tables among servers, but this could be a challenge operation when doing relations between tables. For this reason we opted for using a NoSQL database. Many of NoSQL databases are built having the scalability functionality from start.

One option for a NoSQL dabase is MongoDB, there are others but this one seems to have popularity, it is free and well documented. There are no strong reasons for excluding options like Apache Cassandra or others.

The authentication problem that XMPP raised when implementing a client on a web browser is easily solved by using cookies to identify authenticated users. A cookie is mantained as long as is specified by its expiration date and the user only needs to reauthenticate when it logs out or the cookie expires, changing from page to page does not raises any reauthentication problem.

On our solution the actions that are performed by clients are directly sent to web servers that control the validity of the actions and performs the respective changes to the database. This logic on the XMPP approach would be distributed among XMPP servers and not the webservers, so now our webservers will increase their complexity by implementing the features we needed from XMPP.

