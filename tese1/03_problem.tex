\subsection{Problem Statement} % English

	As communications technologies appeared, we adapted the way we communicate. This project doesn't aims to replace the current video and audio communications, but to enrich them with hyper-media content and make them a more natural and easy to learn process. 

	{\color{blue}With the advent of WebRTC, it became possible to develop video conference web apllications without plugins, this presents a range of possibilities on what can be implemented using already existing web technologies.}
		
    Real-time communication applications can make a difference on business, education and health sectors by providing tools for teaching and learning online, teamworking and socializing.

		For multiple reasons, we often need to repeat or postpone some of our tasks, some people tend to forget what they ear or see.
               %RP acho que toda a gente se esquece eventualmente. Pode não lhe parecer importante na altura. Não é preciso estar doente!
        A real-time system is a huge source of information that requires much attention from its users. An application that provides a way to remember our past communications would be a strong tool for not only to catch what we lost but also to enhance our knowledge.
%RP estes dois ultimos pontos são melhores. Não fales em pessoas com problemas de memoria mas em termos de lidar com grandes quantidades de informação, realizarmos comunicações em grupo (ainda nãofalaste diss) e nem sempre ser possível estar presente.
        %RP também há outros mercados, como online learning, projectos de longa duração que vão evoluindo e onde é preciso revisitar o que aconteceu antes. Fins histórios. Necessidade de manter registos devido a requisitos legais/registar história.

		

                %RP esta última frase está confusa.

        %Estás muito parco nos objectivos do projecto. Acho que falta dizer que é um sistema para complementar/subsitutir actuais sistemas de comunicação audio/video real-time, adicionando-lhe a capacidade de: adicionar conteúdo posteriomente (e.g. anotações) por forma a aumentar o seu valor; permitir acesso posterior (mesmo que apenas 5 minutos depois para quem chega atrasado); permitir navegar na comunicação (hypermedia, saltos, fast-forward, rewind).
