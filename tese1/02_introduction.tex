\section{Introduction}

\subsection{Context}   % English
Since the early days of Human History, we tried to communicate over far locations, from smoke signals to letters delivered by messengers. Real-time communications were limited or even nonexistent. Despite all the efforts made to improve communications, written communication could never replace face to face communication.
With the advent of the telephone network, communications have taken a very important step for us to feel more connected with whom we communicate. Still, only the human voice was not enough, and the invention of cameras and consequent video digitization were a huge step for real-time communications.

	In the past, handwritten documents were limited to a writer per page at a time. Writing a book collaboratively was a difficult task due to synchronism between writers.
	Today, we can achieve more, it is possible to write a document collaboratively, correct spelling mistakes without wasting paper, restructure text at any moment, add a video to a newspaper article and more. Although much of what was said seems banal nowadays, none of this was possible before the computer's invention. 

	As Martin Geddes states\cite{geddes}, \say{No computer in our lifetimes will ever rival a human voice's capacity to conveying rich and complex social and emotional meaning} , although nothing replaces the physical contact with a person while we communicate, we are at a time when we can do more than just a visual and verbal communication, hypermedia can be added to video and voice in order to extend its value. The concept of structured voice and video synchronized with hypermedia is called hypervoice.{\color{red}\textbf{[cite!]}}
        %RP a citação do geddes faz sentido? Parece estar a falar de voz gerada por computador e não de video conferencia onde o que ouvimos são outras pessoas!
        




