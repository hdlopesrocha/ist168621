\chapter{Introduction}
\label{chapter:introduction}
%The introduction is certainly the most read section of any deliverable, and it largely determines the attitude of the reader/reviewer will have toward the work. Therefore, it is probably the most delicate part of the writing of a report.

\section{Background}
\label{section:background}
%Nesta secção devem descrever a área em que a vossa tese se insere, de forma a contextualizar o problema que vão resolver. No âmbito da área devem identificar os principais problemas que existem.

	Since the early days of Human History, we tried to communicate over far locations, from smoke signals to letters delivered by messengers. Real-time communications were limited or even nonexistent. Despite all the efforts made to improve communications, written communication could never replace face to face communication.
	With the advent of the telephone network, communications have taken a very important step for us to feel more connected with whom we communicate. Still, only the human voice was not enough, and the invention of cameras and consequent video digitization were a huge step for real-time communications.

	In the past, handwritten documents were limited to a writer per page at a time. Writing a book collaboratively was a difficult task due to synchronism between writers.
	Today, we can achieve more, it is possible to write a document collaboratively, correct spelling mistakes without wasting paper, restructure text at any moment, add a video to a newspaper article and more. Although much of what was said seems banal nowadays, none of this was possible before the computer's invention. 

	As Martin Geddes states\cite{geddes}, \say{No computer in our lifetimes will ever rival a human voice's capacity to conveying rich and complex social and emotional meaning} , although nothing replaces the physical contact with a person while we communicate, we are at a time when we can do more than just a visual and verbal communication, hypermedia can be added to video and voice in order to extend its value. The concept of structured voice and video synchronized with hypermedia is called hypervoice\cite{geddes}.


\section{Proposed Solution}
\label{section:proposed}


	As communications technologies appeared, we adapted the way we communicate. This project doesn't aims to replace the current video and audio communications, but to enrich them with hyper-media content and make them a more natural and easy to learn process. 

	{\color{blue}With the advent of WebRTC, it became possible to develop video conference web apllications without plugins, this presents a range of possibilities on what can be implemented using already existing web technologies.}
		
    Real-time communication applications can make a difference on business, education and health sectors by providing tools for teaching and learning online, teamworking and socializing.

	For multiple reasons, we often need to repeat or postpone some of our tasks, some people tend to forget what they ear or see.
               %RP acho que toda a gente se esquece eventualmente. Pode não lhe parecer importante na altura. Não é preciso estar doente!
	A real-time system is a huge source of information that requires much attention from its users. An application that provides a way to remember our past communications would be a strong tool for not only to catch what we lost but also to enhance our knowledge.





\section{Thesis Contribution}
\label{section:contribution}
%Nesta secção devem identificar como é que a vossa solução vai contribuir para resolver o problema.

	Making it clear, this project aims to complement current audio, text and video communications in order to create rich and collaborative interfaces with the ability to add more content on a future time (e.g. creating annotations for improving content search) in order to increase its value. Another important goal of this project is the ability to navigate in time by rewinding communications, fast-forward and jump to certain points.

	A web application with an easy to learn user interface will be developed to accomplish solving our problem. Our application, unnamed yet, is targeted at web browsers that are compatible with only standard technologies like JavaScript, \ac{WebRTC}, \ac{HTML}5 and \ac{CSS}3. Any additional plug-in is avoidable, \emph{JavaScript} libraries will be preferred as they can be downloaded on the fly.  

	Although we propose an initial solution to solve our problem, we will make a continuous effort to survey the systems that solve our problem either completely or just a part of it, not excluding solutions that may appear during the implementation of our project.

	We will present an architecture that can meet our goals, implement the respective prototype and test it with real users, unitary tests and benchmarks.

	According to Martin Geddes, the quality of the interaction worsens as the number of users increase\cite{geddes}. In our testing phases we will quantify and qualify the impact of increasing users on the interface and performance of our prototype. 

	All the problems faced during the development and limitations will be reported on the thesis so that a future project better then ours can be easily and better developed.



\section{Outline}

This document is structured as follows:

\begin{itemize}
\item \textbf{Chapter \ref{chapter:introduction}} presents the motivation, background and proposed solution.
\item \textbf{Chapter \ref{chapter:relatedwork}} describes the previous work in the field.
\item \textbf{Chapter \ref{chapter:architecture}} describes the system requirements and the architecture for an Web Application that fulfills the goals of this thesis.
\item \textbf{Chapter \ref{chapter:implementation}} describes the implementation of our Web Application and the technologies chosen.
\item \textbf{Chapter \ref{chapter:evaluation}} describes the evaluation tests performed and the corresponding results.
\item \textbf{Chapter \ref{chapter:conclusion}} summarizes the work developed and future work.
\end{itemize}

\cleardoublepage
