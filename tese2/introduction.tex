\chapter{Introduction}
\label{chapter:introduction}
%The introduction is certainly the most read section of any deliverable, and it largely determines the attitude of the reader/reviewer will have toward the work. Therefore, it is probably the most delicate part of the writing of a report.

\section{Background}
\label{section:background}
%Nesta secção devem descrever a área em que a vossa tese se insere, de forma a contextualizar o problema que vão resolver. No âmbito da área devem identificar os principais problemas que existem.

	Since the early days of Human History, we tried to communicate over far locations, from smoke signals to letters delivered by messengers. Real-time communications were limited or even nonexistent. Despite all the efforts made to improve communications, written communication could never replace face to face communication.
	With the advent of the telephone network, communications have taken a very important step for us to feel more connected with whom we communicate. Still, only the human voice was not enough, and the invention of cameras and consequent video digitalization were a huge step for real-time communications.

	In the past, handwritten documents were limited to a writer per page at a time. Writing a book collaboratively was a difficult task due to synchronization between writers.
	Today, we can achieve more. It is possible to write a document collaboratively, correct spelling mistakes without wasting physical resources, restructure text at any moment, add a video to a newspaper article and more. Although much of what was said seems banal nowadays, none of this was possible before the computer's invention. 

	As Martin Geddes states\cite{geddes}, \say{No computer in our lifetimes will ever rival a human voice's capacity to conveying rich and complex social and emotional meaning}. Although nothing replaces the physical contact with a person while we communicate, we are at a time when we can do more than just visual and verbal communication. Hypermedia can be added to video and voice in order to extend its value. The concept of structured voice and video synchronized with hypermedia is called hypervoice\cite{geddes}.

	As communications technologies appeared, we adapted the way we communicate. The purpose of this project is not the replacement of the current video and audio communications, but to enrich them with hyper-media content and make them a more natural and easy to learn process. 

	With the advent of WebRTC and its successive integration with web browsers, it became possible to develop video conference web applications without plugins. This presents a range of possibilities on what can be implemented using already existing web technologies.
		
    Furthermore, real-time communication applications can make a significant difference on business, education and health sectors by providing tools for developing teaching and learning online, teamworking and socializing web applications.

        %RP Parece-me que até aqui ainda é background ou motivação. Não me parece que estejas já a descrever a solução proposta.


\section{Proposed Solution}
\label{section:proposed}

	For multiple reasons, we often need to repeat or postpone some of our tasks, some people tend to forget what they hear or see.     
        
	Therefore, a real-time system is a huge source of information that requires much attention from its users. In this context, an application that provides a way to remember our past communications would be a strong tool for not only to catch what we lost but also to enhance our knowledge.

	Our goal in this project is to develop an application targeted to the web platform, resorting to \ac{WebRTC}, that leverages the hyper-linked communications by providing a video conference environment enriched with interactive and non-interactive discrete media types such as images, subtitles, forms and all types of content that can be added using \ac{HTML}5, \ac{CSS}3 and \emph{JavaScript} including continuous media types such as video, music and animations.

	One of the key features of this project is the ability to navigate in time in order to reproduce the conversation again or introduce hyper-content to it such as time annotations, interactive lists of topics and subtitles. In this context we also provide a simpler method for creating and synchronizing hyper-content using \ac{QR} codes.

	In addition to this conference environment, which provides different functionalities than traditional conference environments such as \emph{Skype} and \emph{Google Hangouts}, we also enable a collaborative text editor and a chat that supports sending time hyper-links and files to conference participants.

	Furthermore, another relevant feature is the possibility to compose multiple video streams into a single one, which enables adding more users to conference rooms without impacting on clients performance. Users can change to individual streams on demand or automatically to the talking users.
        

\section{Thesis Contribution}
\label{section:contribution}
%Nesta secção devem identificar como é que a vossa solução vai contribuir para resolver o problema.

Making it clear, this project aims to complement current audio, text and video communications in order to create rich and collaborative interfaces with the ability to add more content on a future time (\emph{e.g.} creating time annotations for improving content search) in order to increase its value. It is also important to highlight another goal of this project which is the ability to navigate in time by rewinding communications, fast-forward and jump to certain points.

%RP Estás a misturar goals com contribuições. Nesta altura deves falar do que foi feito (na solução). Os objectivos podem ficar no inicio da arquitectura para a motivar e condicionar 

A web application with an easy to learn user interface was developed to accomplish solving our problem. Our application, unnamed yet, is targeted at web browsers that are compatible with only standard technologies like JavaScript, \ac{WebRTC}, \ac{HTML}5 and \ac{CSS}3. Any additional plug-in was avoided, \emph{JavaScript} libraries were preferred as they can be downloaded on-the-fly.

%RP ``will be developed''!!! ``was''. Não podes estar a falar no futuro. Isto é passado. Estás a descrever o que fizeste.

	We have presented an architecture that can meet our goals, implemented the respective prototype and tested it with real users and performance benchmarks.

        %RP mais uma vez, passado e muito vago.
        
	According to Martin Geddes, the quality of the interaction worsens as the number of users increase\cite{geddes}. In our testing phases we will quantify and qualify the impact of increasing users on the interface and performance of our prototype. 

	All the problems faced during the development and limitations were reported on the thesis so that a future project better then ours can be easily and better developed.

        %RP este capítulo precisa de uma revisão. A estrutura tem de ser:
        %Motivação/background - descrever os problemas / desafios / oportunidades que motivam a tese
        %Proposed solution - descrever por alto a solução para o problema, uma aplicação web com determinadas funcionalides (listando todas elas)
        %contribution - as contribuições são: uma arquitectura, um interface utilizador para lidar com um conceito novo (andar no tempo), um protótipo funcional, teste com utilizadores e respectivas conclusões (avaliação experimental), documentação (tese).

\section{Outline}

This rest of this document is structured as follows:

\begin{itemize}
%\item \textbf{Chapter \ref{chapter:introduction}} presents the motivation, background and proposed solution.
\item \textbf{Chapter \ref{chapter:relatedwork}} describes the previous work in the field.
\item \textbf{Chapter \ref{chapter:architecture}} describes the system requirements and the architecture for an Web Application that fulfills the goals of this thesis.
\item \textbf{Chapter \ref{chapter:implementation}} describes the implementation of our Web Application and the technologies chosen.
\item \textbf{Chapter \ref{chapter:evaluation}} presents the evaluation tests performed and the corresponding results.
\item \textbf{Chapter \ref{chapter:conclusion}} summarizes the work developed and proposes future work.
\end{itemize}

\cleardoublepage
