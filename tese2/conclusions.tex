\chapter{Conclusions}
\label{chapter:conclusion}

\section{Summary}
\label{section:summary}
Neste secção deve-se fazer o resumo do trabalho efectuado, retomando a ideia, as
contribuições definidas e a forma como estas se materializaram

In this thesis, our goal was the development of a web application using \emph{WebRTC} that complement current audio, text and video communications in order to create rich and collaborative interfaces with the ability to add more content on a future time. Another important goal of this project is the ability to navigate in time by rewinding communications, fast-forward and jump to certain points.

Having this in mind, we started to analyze the technologies necessary to implement our solution. In a first glance, we have analyzed the problems that real time communication applications are facing when using the current Internet infrastructure. Next we have we have analyzed the main components of the \emph{WebRTC} stack. Furthermore, we have analyzed how we could implement the signalization component that \emph{WebRTC} does not not define. Hence, with the transport component being analyzed, we have studied how we could expose content in a synchronized and interactive way. Moreover, we have performed an analysis on the kinds of media types the particularities of each one when manipulating time. Lastly, on the state of the art context, we have analyzed existing frameworks and libraries that could help us implementing a collaborative environment.

Based on our evaluation of the state of the art, we have defined the architecture of our proposed solution.

Although we have used our own signaling protocol on our implementation, the first steps we have made during the development of our prototype were using \ac{XMPP}, we have discovered the drawbacks of this approach in practice and we moved our learnings to the state of the art. Besides starting our prototype using this approach, we have used \emph{strophe.js} for performing communications with \ac{XMPP} servers. We have found and fixed a bug present on \emph{strophe.js} library that silenced an error raised by duplicated user registrations.

Another important contribution we have made was a bug fix of a memory leak present on \ac{KMS} which we discovered when we were doing performance tests with our solution.

\section{Achievements}
\label{section:achievements}
	We have successfully implemented the basic functionalities of our prototype and spare some time for adding more valuable features such as the ability to create content by exposing \emph{QR codes} to the camera and lastly perform changes to the user interface in order to improve the quality of user's experience.

	The performance tests that we have executed showed that our system is stable and more importantly that our web server is lightweight and most of processing power is dedicated to the streaming server.

	Our usability tests show results that are considerably worse than the established optimal values due to our solution propose a different way to communicate that most people are not used to. Although we have obtained those results, in general our users gave us positive feedback and valuable advices which we have used to improve our system. 

\section{Future Work}
\label{section:future}
Nesta secção devem identificar o trabalho futuro, sob duas perspectivas:
\begin{itemize}
\item o trabalho que resulta directamente dos problemas que a vossa proposta criou, ou não conseguiu resolver
\item o trabalho que resulta da evolução do sistemas
\end{itemize}


\cleardoublepage
