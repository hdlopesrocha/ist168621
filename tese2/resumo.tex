\chapter*{Resumo}

% DO NOT CHANGE THIS - Add entry in the table of contents as chapter
\addcontentsline{toc}{chapter}{Resumo}

O conceito de comunicações hiper-ligadas aplica muitos dos conceitos de hiper-media, bastante utilizados no conteúdo \emph{Web}. Este paradigma permite sincronizar, estruturar e navegar sobre conteúdo proveniente de communicações integrado em chamadas de voz e vídeo.


%A voz e a imagem justas exprimem emoções como nenhum outro tipo de media permite. Através dos conceitos de hiper-media, podemos adicionar mais valor às comunicações por vídeo-conferência.

A tecnologia \ac{WebRTC} permite realizar comunicações em tempo real entre navegadores web sem a necessidade de instalar \emph{sofware} adicional. A natureza das aplicações \emph{web} já usufruem dos conceitos de hiper-media, o que faz do \ac{WebRTC} a tecnologia ideal para aplicar o conceito de comunicações hiper-ligadas.
A plataforma providenciada por navegadores \emph{web} disponibiliza uma camada de abstração que torna possível correr aplicações independentemente do sistema operativo.
O suporte nativo do \ac{WebRTC} nos sistemas operativos extendem a sua utilização para fora do contexto dos navegadores \emph{web}, possibilitando explorar funcionalidades que os navegadores \emph{web} suportam de uma forma limitada como a gravação de vídeo e armazenamento de informação em massa.

O nosso objectivo neste projecto é desenvolver uma aplicação, recorrendo ao \ac{WebRTC}, que utiliza o conceito de comunicações hiper-ligadas de forma a enriquecer as comunicações em conferência com vários tipos de media, editores colaborativos, anotações temporais, envio de mensagens, possilidade de sobrepor contéudo ao vídeo e voltar atrás no tempo.


Neste documento, apresentamos o actual Estado de Arte nas comunicações hiper-ligadas e as tecnologias relacionadas, propomos e implementamos uma arquitectura para uma aplicação com comunicações hiper-ligadas baseada no \emph{WebRTC}. O nosso trabalho foi avaliado por utilizadores que, para além de terem gostado da experiência com a nossa aplicação, acharam o nosso protótipo bastante inovador.

\vspace{1cm}

% TODO 4 to 6 keywords;
\textbf{\Large Palavras-chave:} WebRTC, comunicações asíncronas, ferramentas colaborativas, navegação temporal, vídeo-conferência não linear, gravação vídeo


\cleardoublepage
