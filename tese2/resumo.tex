\chapter*{Resumo}

% DO NOT CHANGE THIS - Add entry in the table of contents as chapter
\addcontentsline{toc}{chapter}{Resumo}

O conceito de comunicações hiper-ligadas aplica muitos dos conceitos de hiper-media, bastante utilizados no conteúdo \emph{Web}, para serviços se comunicação e colaboração. Este paradigma permite sincronizar, estruturar e navegar sobre conteúdo proveniente de communicações, colocando ao dispor formas de integrar conteúdo de mídia social e ferramentas colaborativas em chamadas de voz e vídeo.
A voz e a imagem justas exprimem emoções e expoem a creatividade como nenhum outro tipo de media permite. Através dos conceitos de hiper-media, podemos adicionar mais valor às comunicações por voz e vídeo.

A tecnologia \ac{WebRTC} permite realizar comunicações em tempo real entre navegadores web sem a necessidade de instalar aplicações adicionais ou \emph{plug-ins}. A natureza das aplicações \emph{web} já usufruem dos conceitos de hiper-media, o que faz do \ac{WebRTC} a tecnologia ideal para aplicar o conceito de comunicações hiper-ligadas.
A plataforma providenciada por navegadores \emph{web} disponibiliza uma camada de abstração que torna possível correr aplicações independentemente do sistema operativo.
O suporte nativo do \ac{WebRTC} nos sistemas operativos extendem a sua utilização para fora do contexto dos navegadores \emph{web}, possibilitando explorar funcionalidades que os navegadores \emph{web} suportam de uma forma limitada como a gravação de vídeo e armazenamento de informação em massa.

O nosso objectivo neste projecto é o desenvolvimento de uma aplicação que utiliza o conceito de comunicações hiper-ligadas de forma a enriquecer ambientes colaborativos e comunicações audio e vídeo com vários tipos de media, que podem ser acedidos em tempo real ou mais tarde.
Esta aplicação será desenvolvida sobre a plataforma \emph{web}, recorrendo ao \ac{WebRTC}.

Neste documento, apresentamos o actual Estado de Arte nas comunicações hiper-ligadas e as tecnologias relacionadas, propomos e implementamos uma arquitectura para uma aplicação com comunicações hiper-ligadas baseada no \emph{WebRTC} e mostramos os resultados da implementação e avaliação.

\vspace{1cm}

% TODO 4 to 6 keywords;
\textbf{\Large Palavras-chave:} WebRTC, comunicações, asíncronas, colaboração


\cleardoublepage
