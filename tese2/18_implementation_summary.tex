\section{Chapter Summary}
\label{implementation:summary}

In this chapter we have described how we implemented the various components present on our architecture, the challenges that we have faced and the solutions we have found.

We started by defining the database model, which influenced directly our system's behavior and the extensibility to new functionalities.

Then, we underlined the requirements of our signaling protocol and, as a consequence, we have described, in detail, the protocol itself.

With the signaling protocol implemented, we observed the first outcomes of using \ac{WebRTC}, the basic functionality we have implemented was an echo of the video and audio sent by users. However we have implemented other features such as switching to other user streams, recording and mixing multiple streams into a single one.

In respect to hyper-content, we have described how to create and search for content either being superimposed content to video and time annotations. In relation to displaying content to users we have also defined how we have synchronized the content to show and the security concerns of our choices.

Then, we have described how we have implemented our chat and collaborative editor.

