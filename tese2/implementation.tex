\chapter{Implementation}
\label{chapter:implementation}

%\section{Implementation Options}
%Neste secção devem apresentar as opções de implementação que tinham ao vosso
%dispor, avaliá-las e justificar a escolha que realizaram. Isto pode englobar:
%\begin{itemize}
%    \item Simuladores
%    \item Linguagens e ambientes de programação
%    \item Sistemas operativos
%    \item Hardware
%\end{itemize}
%Fica sempre bem na avaliação colocar uma tabela com as características
%pretendidas e as que são satisfeitas pelas várias opções (tipo catálogo com as
%características dos automóveis). A escolha deve surgir naturalmente, com base na
%opção que tem mais cruzes\ldots

%\section{Architecture}
%Nesta secção devem explicar como implementaram a vossa solução, apresentando
%as simplificações que efectuaram, face ao modelo inicialmente previsto. As
%simplificações devem ser devidamente justificadas. Se for possível, devem indicar
%que estas não põem em causa as contribuições da tese.
%Podem ainda descrever os principais problemas que tiveram e a forma como os
%abordaram e resolveram.

%Se estiverem a usar um simulador devem:

%\begin{itemize}
%    \item explicar o funcionamento do simulador
%    \item explicar as alterações e modelos que desenvolveram no simulador e que permitem validar a vossa ideia
%\end{itemize}

%Se estiveram a desenvolver SW, sem simulador devem:
%\begin{itemize}
%    \item explicar os módulos, interfaces, estruturas de dados, etc\ldots
%\end{itemize}

%Sempre que possível, ilustrem a arquitectura com figuras que demonstrem a
%evolução face à arquitectura da secção anterior. Isto é, usem as figuras anteriores
%e façam as modificações necessárias à obtenção da arquitectura do protótipo.
%\ldots

In this chapter we are going to present our implementation choices, the difficulties that we have faced and the strategies available to overcome them.

To this end, we are going to present and explain our database model, how we have improved it and how this model can help us reach our goals. 

Subsequently, we are going to describe our signaling protocol which will make it possible to use the \ac{WebRTC}'s functionalities in order to implement our system's streaming features.

With the model and signaling protocol defined we are prepared to show the multiple approaches we have followed in order to implement stream recording, their drawbacks and our choices.

Then are going to describe how we show hyper-content to users, namely the multiple ways to create hyper-content and the algorithm behind the user interface synchronization, the security flaws of our content displaying mechanism and a possible solution to overcome vulnerabilities. 

Not less important we also are going to describe how we have implemented our application time-line and the functionalities above it such as time manipulation and annotations management. Among other different functionalities we also are going to describe how we perform stream composition, synchronization of our collaborative text editor and implementation of our instant messenger.

Lastly, we are going to describe the deployment of our solution in respect to the hardware that we have used and the software installed onto it.

\section{Model}
The model is the most important component of our solution, a badly designed model can imply serious difficulties when implementing new features that are not part of the plans, sometimes we had to redesign the model in order to support new features.

\subsection {Schema representation}
    \emph{MongoDB} has a slightly different terminology from relational databases. The first big difference is instead of having tables \emph{MongoDB} stores its objects on collections. The analogous data structure to the table row is a document.

    Each \emph{MongoDB} document is represented by a \emph{JSON} object and, as a result, each document may have different attributes within a collection. Needless to say that following this approach we don't need to create the collections with a well predefined schema, in fact we don't to define it at all. Although, for reasons of coherence and organization, we represent our database collections as they would have a predefined schema by following the same document structure.

    Similarly to relational databases, \emph{MongoDB} requires a primary key (typically typed as \emph{ObjectId} and named \emph{"\_id"}) for each document which is automatically assigned if not specified. 

    In order to define \emph{foreign keys}, we just store them as \emph{ObjectId}s if and only if the foreign keys point to documents within a unique collection, otherwise we need an additional attribute to specify which collection is the \emph{foreign key} pointing at.

    In respect to attributes nullability, \emph{MongoDB} does not enforce a document's attribute to have a non nullable value, although, for sake of good functionality, we perform those constraints validation programmatically and by so we also represent them in our schema.

    An example of schema representation can be seen on table \ref{table:schema}.

\begin{table}[!htb]
\centering
\caption{Schema representation}
\label{table:schema}
    \begin{tabular}{|ll|}
        \hline
        \multicolumn{2}{|c|}{\textbf{Collection name}}              \\ \hline
        $\Diamondblack$ \underline{\_id (primary key)}  & ObjectId  \\ 
        $\medbullet$ Not nullable property name  & Property type    \\ 
        $\medcirc$ Nullable property name      & Property type      \\
        $\medcirc$ Reference to document     & ObjectId             \\
        $\medbullet$ Embedded document      & Document              \\ 
        $\medbullet$ Embedded list      & List[Type]                \\ \hline
    \end{tabular}
\end{table}

\subsection {Generic model}

For designing our model we have taken into account generic programming techniques. We observed that operations like searching for an object was quite repeated amongst different types of objects. 

Our first decision for our model in order to avoid repeated code, was the isolation of the object's attributes from themselves, so we could apply the search operation to a set of attributes independently from the object type. To this generic set of properties we call data (table \ref{table:generic}) and each object of this type has a reference to the owner, which is a unique identification number.

\begin{table}[!htb]
\centering
\caption{Generic data model}
\label{table:generic}
    \begin{tabular}{cc}
        \begin{tabular}{|ll|}
            \hline
            \multicolumn{2}{|c|}{\textbf{Data}}            \\ \hline
            $\Diamondblack$ \underline{\_id}  & ObjectId     \\ 
            $\medbullet$ owner           & ObjectId        \\ 
            $\medbullet$ properties      & List[Attribute] \\ 
            $\medcirc$ searchableValues & List[Text]       \\ \hline
        \end{tabular}
        \begin{tabular}{|ll|}
            \hline
            \multicolumn{2}{|c|}{\textbf{Attribute}}       \\ \hline
            $\medbullet$ key           & Text              \\ 
            $\medcirc$ value           & Object            \\ 
            $\medbullet$ identifiable    & Boolean         \\ 
            $\medcirc$ readPermissions &  List[ObjectId]   \\ 
            $\medcirc$ writePermissions  & List[ObjectId]  \\ \hline
        \end{tabular}    
    \end{tabular}
\end{table}

The identification number by itself is not sufficient to identify an object, objects from different types can have the same identification number. In order to solve this problem, when an object is created, its correspondent data must contain the owner's object type. 

Whenever an attribute is created, it must be specified the attributes name, its value, searcheability, identifiability and the set of object identification numbers that can read and write that attribute. 

In relation to our permission mechanism, if the read or write sets are not specified we assume the attribute is readable and writable by everyone, conversely if the read and write sets are empty, nobody is allowed to read or write the attribute. Implicitly if an entity can write an attribute it can also read it.

In particular, if all attributes were searchable it could be simple to search for attributes that could reveal sensible information about an object. For example if we consider that a user could have a health related attribute, searching by a disease would reveal which users could suffer from a certain disease, the leak of that kind of information could, for instance, change the agreement between users and health insurance companies. For this reason only the specified attributes as searchable will be taken into account when performing keyword searches.

Another important attribute specification is the owner identifiability, which tells us if the attribute identifies the object. This specification let us create abstract authentication services, for example a user can login into our system by providing any attribute that identifies himself, for example the e-mail but others are possible like the user name or cellphone number. 

Not less important, in order to get an object properties efficiently, we have created an index over the \emph{owner} attribute. In the meantime , we have also created an index over the \emph{searchableValues} in order to improve the keyword search performance.

In summary, with this model we can perform search and identification of any kind of objects, as we will see on the following models, the \emph{user} and \emph{group} models are using this generic model for storing their attributes. 
% Not less important, attributes can specify aggregations of objects. For example, the user role is an aggregator property that within users it allows the identifaction of each one is administrator. This aggregation specification is independent from the object type, so it's possible to search for the administrative role and return users and groups of users that contains that property. <= [currently this operation is possible but it is not specified]


\subsection{User model}

The user model is not tied to the user attributes (table \ref{table:user}), the information maintained in this model is just used for authentication purposes. Passwords are not stored in plain text, instead we apply hashing and salting techniques \cite{password} in order to make it harder to decode the password by an attacker. Accordingly, we use \emph{SHA-1} and a random salt per user with 32 characters long.

\begin{table}[!htb]
\centering
\caption{User model}
\label{table:user}
    \begin{tabular}{|ll|}
        \hline
        \multicolumn{2}{|c|}{\textbf{User}}         \\ \hline
        $\Diamondblack$ \underline{\_id}  & ObjectId  \\ 
        $\medbullet$ hash           & Text          \\ 
        $\medbullet$ salt      & Text               \\ \hline
    \end{tabular}
\end{table}

\subsection{Relation model}

A relation between two entities $e_1$ and $e_2$ is represented by the pair $e_1\rightarrow e_2$ (table \ref{table:relation}), where $e_1$ is the source and $e_2$ is the target. This relation is said bi-directional if and only if it also exists the relation $e_2\rightarrow e_1$.

\begin{table}[!htb]
\centering
\caption{Relation model}
\label{table:relation}
    \begin{tabular}{|ll|}
        \hline
        \multicolumn{2}{|c|}{\textbf{Relation}}     \\ \hline
        $\Diamondblack$ \underline{\_id}  & ObjectId  \\ 
        $\medbullet$ source           & ObjectId    \\ 
        $\medbullet$ target      & ObjectId         \\ \hline
    \end{tabular}
\end{table}

A user can only interact with friends or with group members. In order to validate a friendship, both users must agree on that friendship, by other words it must exist a bi-directional relation between both users.

In order to improve performance of queries over \emph{Relation} collection we have created indexes on the \emph{source}, \emph{target} and also on the pair composed by both attributes.

\subsection{Group model}

A group is composed by an \emph{id}, \emph{inviteToken} and a \emph{visibility} as shown on table \ref{table:group}.

Moreover, a group can be public or private. If the group is public then it is visible to all users that maintain a friendship with a member of this group. If the group is private then it is only visible to their members.

\begin{table}[!htb]
\centering
\caption{Group model}
\label{table:group}
    \begin{tabular}{cc}
        \begin{tabular}{|ll|}
            \hline
            \multicolumn{2}{|c|}{\textbf{Group}}        \\ \hline
            $\Diamondblack$ \underline{\_id}  & ObjectId  \\ 
            $\medcirc$ inviteToken           & Text     \\ 
            $\medbullet$ visibility      & Text         \\ \hline
        \end{tabular}

        \begin{tabular}{|ll|}
            \hline
            \multicolumn{2}{|c|}{\textbf{GroupMembership}}  \\ \hline
            $\Diamondblack$ \underline{\_id}  & ObjectId      \\ 
            $\medbullet$ groupId            & ObjectId      \\ 
            $\medbullet$ userId & ObjectId                  \\ \hline
        \end{tabular}    
    \end{tabular}
\end{table}


The group membership is a special case of relation, where the target entity is always a group.

When a group is created, a group membership is automatically assigned by its creator.

Entities that have a membership with a group can create more memberships by sharing an invite token or by specifying new group members.

In order to improve performance of queries over \emph{GroupMembership} collection we have created indexes on the \emph{groupId}, \emph{userId} and also on the pair composed by both attributes.


\subsection{Message model}

A message is composed by its content, time of creation and source and target identification numbers (table \ref{table:message}). The message's target can reference any of object but our application is only covering messages to groups.

\begin{table}[!htb]
\centering
\caption{Message model}
\label{table:message}
    \begin{tabular}{|ll|}
        \hline
        \multicolumn{2}{|c|}{\textbf{Message}}      \\ \hline
        $\Diamondblack$ \underline{\_id}  & ObjectId  \\ 
        $\medbullet$ source           & ObjectId    \\ 
        $\medbullet$ target      & ObjectId         \\ 
        $\medbullet$ content      & Text            \\ \hline
    \end{tabular}
\end{table}

In order to improve performance of queries over \emph{Relation} collection we have created indexes on the \emph{target} attribute and also on the pair composed by \emph{target} and \emph{\_id} attributes for finding and sorting the messages received by an entity more efficiently.

\subsection{Hyper content model}

During a group conversation it is possible to create time annotations for making it easy to access that time either by searching or sharing with other users.

A time annotation (table \ref{table:hyper}) contains a title, the correspondent group identification number and the time itself.

The hyper content is used to synchronize content among users during a conversation. Table \ref{table:hyper} shows that every hyper content must have a start and ending time, the correspondent group identification number and the content itself in the form of text. Beside those attributes, in order to perform queries over the \ac{HTML} contents with more precision, we have added an additional \emph{searchableContent} attribute that contains just the searchable content extracted from the content excluding the \ac{HTML} tags and parameters using \emph{Jsoup}\footnote{\url{http://jsoup.org/} (Accessed April 11, 2016)}. 


\begin{table}[!htb]
\centering
\caption{Hyper content model}
\label{table:hyper}
    \begin{tabular}{cc}
        \begin{tabular}{|ll|}
            \hline
            \multicolumn{2}{|c|}{\textbf{HyperContent}}  \\ \hline
            $\Diamondblack$ \underline{\_id}  & ObjectId \\ 
            $\medbullet$ groupId            & ObjectId   \\ 
            $\medbullet$ start              & Date       \\ 
            $\medbullet$ end                & Date       \\ 
            $\medbullet$ content            & Text       \\ 
            $\medbullet$ searchableContent  & Text       \\ \hline
        \end{tabular}

        \begin{tabular}{|ll|}
            \hline
            \multicolumn{2}{|c|}{\textbf{TimeAnnotation}}  \\ \hline
            $\Diamondblack$ \underline{\_id}  & ObjectId   \\ 
            $\medbullet$ groupId            & ObjectId     \\ 
            $\medbullet$ title & Text                      \\ 
            $\medbullet$ time & Date                       \\ \hline

        \end{tabular}    
    \end{tabular}
\end{table}

For example the content showed in listing \ref{lst:jsoup} would produce the searchable content "Click here!".

\begin{minipage}{\linewidth}
\begin{lstlisting}[caption={Example of HTML content},label={lst:jsoup},language=JavaScript]
<div class="subtitle">
    <a href="/">Click here!</a>
</div>
\end{lstlisting}
\end{minipage}

In order to improve the performance of searching content and calculate interval intersections, we have created indexes over the following sequences of attributes: 

\begin{itemize}
\item{\emph{\{groupId,start,end\}} $\rightarrow$ finding visible contents for a given time  within a group (intersections).}
\item{\emph{\{groupId,start\}} $\rightarrow$  finding contents that starts after a given time within a group (for pre-loading).}
\item{\emph{\{groupId,searchableContent\}} $\rightarrow$ searching contents by keywords within a group.}  
\end{itemize}

\subsection{Collaborative Content model}

Within a conversation, users can write documents collaboratively. As we can see in table \ref{table:collaborative}, each document has a content and a reference to the correspondent group. 


\begin{table}[!htb]
\centering
    \caption{Collaborative content model}
    \label{table:collaborative}
    \begin{tabular}{|ll|}
        \hline
        \multicolumn{2}{|c|}{\textbf{CollaborativeContent}}  \\ \hline
        $\Diamondblack$ \underline{\_id}  & ObjectId           \\ 
        $\medbullet$ groupId            & ObjectId           \\ 
        $\medcirc$ content            & Text               \\ \hline
    \end{tabular}
\end{table}

In order to improve the performance of finding the group's content we have created and index over the \emph{groupId} attribute.

\subsection{Recording model}

During a conversation, users may allow sharing their web cameras, by doing so their video is stored in recording chunks. Each chunk, as seen in table \ref{table:recording}, represents an interval of time  $T=\big[c^{start},c^{end}\big[$, it contains the media \ac{URL}, a reference to a group, an owner identification number and the correspondent \emph{WebSocket} session id.

In order to allow different devices within the same user, storing just the \ac{URL} with an associated user id is not enough as in this case it is needed one more parameter to differentiate the different \emph{webSocket} sessions opened by the same user. For this reason we had to associate a random session identification number to each \emph{webSocket}.

\begin{table}[!htb]
\centering
    \caption{Recording model}
    \label{table:recording}
    \begin{tabular}{cc}
        \begin{tabular}{|ll|}
            \hline
            \multicolumn{2}{|c|}{\textbf{RecordingInterval}}  \\ \hline
            $\Diamondblack$ \underline{\_id}  & ObjectId        \\ 
            $\medbullet$ groupId            & ObjectId        \\ 
             $\medbullet$ start             & Date            \\ 
            $\medbullet$ end                & Date            \\ \hline
        \end{tabular}   
        \begin{tabular}{|ll|}
            \hline
            \multicolumn{2}{|c|}{\textbf{RecordingChunk}}  \\ \hline
            $\Diamondblack$ \underline{\_id}  & ObjectId     \\ 
            $\medbullet$ groupId            & ObjectId     \\ 
            $\medbullet$ owner           & ObjectId        \\ 
            $\medbullet$ sessionId          & Text         \\
            $\medbullet$ start              & Date         \\ 
            $\medbullet$ end                & Date         \\ 
            $\medbullet$ url               & Text          \\ \hline
        \end{tabular}
    \end{tabular}
\end{table}



A set of chunks $S=[c_1,c_2,\ldots,c_n]$ is said continuous if $\forall c_i\in S, \exists c_j \in S$ where $j\neq i$ and $c_i^{start} = c_j^{end} \vee c_i^{end} = c_j^{start}$. 

A recording interval represents a continuous set of recording chunks.

In order to improve the performance of searching recording chunks and calculate which chunks are are going to be reproduced, we have created indexes over the following sequences of attributes: 
\begin{itemize}
    \item{\emph{\{groupId,start,end\}} $\rightarrow$ finding all available chunks for a given time} 
    \item{\emph{\{groupId,sessionId,end\}} $\rightarrow$ finding intersecting chunks with a given time within a session}
    \item{\emph{\{groupId,owner,end\}} $\rightarrow$ finding intersecting chunks with a given time within a user (for a different session) or group}
    \item{\emph{\{groupId,sessionId,\_id\}} $\rightarrow$ finding chunks that follow a given \emph{id} within a session}
    \item{\emph{\{groupId,owner,\_id\}} $\rightarrow$ finding chunks that follow a given \emph{id} within a user or group}
\end{itemize}

Not less important we have also created an index over the \emph{RecordingInterval}'s \emph{groupId} attribute for listing all intervals withing a conference room. 
\section{Signaling Protocol}

As we have already mentioned on our related work, \ac{WebRTC} does not implement the signaling protocol, which is used to establish connections between peers. This connection may be direct using \ac{STUN} or relayed by a \ac{TURN} server in case of direct communications are not possible.

The signaling protocol plays a fundamental role on our solution. Taking into account the choices we have based on our related work, we have decided to implement our own signaling protocol in order to use \ac{WebRTC} on our solution.

Although we have mentioned that the signaling protocol is used to establish connections between peers, on our system our media server (\ac{KMS}) is a peer that receives video streams and sends to its connected clients. 

In a general view, the signaling protocol is used to share connection and media properties between two peers. In order to ease the understanding of our signaling protocol, we have created a sequence diagram that is represented on Figure \ref{fig:signaling2}\footnote{Although two \ac{ICE} servers are shown they are, in fact, the same. Showing just one \ac{ICE} server would be difficult to draw.}.
%RP devias começar com uma introdução a explicar que tiveste de desenvolver um protocolo para coordenar a interacção dos clientes com o servidor e entre clientes (?).
%RP explicar quando é usado (começar chamada, terminar, juntar, chat(?), edição colaborativa (?), etc).
%HR done


\begin{figure}
    \centering
    \begin{subfigure}{}
    	\includegraphics[width=0.9\textwidth]{figures/signaling}
    \end{subfigure}
    \caption{Signaling sequence diagram}
    \label{fig:signaling2}
\end{figure} 


Before the signaling protocol starts, the user must be authenticated. After the user provide the correct credentials, it receives an \ac{HTTP} \emph{cookie} from our server in order to be identified in the following \ac{HTTP} requests.

%RP faz sentido haver dois ICE? Se queres colocar 2 dá-lhes nomes diferentes (nem que seja cliente e servidor ou 1 e 2).

After the web application server validates the user access, the signaling protocol allows the users to directly connect to the \ac{KMS}, which is placed in a private network, and lets the application server and users negotiate media types and encoding information to use during the conversation.
%RP o it em ``it allows'' é quem? webserver ou signaling protocol?
%HR done
We considered implementing the signaling protocol using \ac{HTTP} messages.
However, they would transport extra information such as \ac{HTTP} headers and would follow a request-response signaling mechanism which would not be the best option, as multiple ICE candidates can arrive at any time. Using a \ac{REST} \ac{API} over \ac{HTTP} for signaling and allowing other methods to be called at the same time would lead to opening multiple \ac{TCP} connections at the same time.

Instead, if we create a \emph{WebSocket} \ac{API}, we only need one \ac{TCP} connection and, at the same, provide bi-directional communications without additional headers. Moreover, using \emph{WebSocket} allows the application server to send messages asynchronously to the client without the client having to request it, which can be slightly faster than using \ac{HTTP} which is based on a request followed by a response.
%RP em vez de slighhtly faster não queres antes enfatizar que pode ser antes o servidor a iniciar a comunicação, de forma assincrona?
%HR done
Our signaling protocol consists of sending and receiving \ac{JSON} formated messages, \emph{e.g.} Listing \ref{lst:msgtype}, over WebSockets by both the application server and the client. 

\begin{minipage}{\linewidth}
\begin{lstlisting}[caption={General structure of our WebSocket messages},label={lst:msgtype},language=json]
{
	"cmd":<cmd>,
	"data":<data>
}
\end{lstlisting}
\end{minipage}

When a user enters a group conference, after the page is completely loaded, a WebSocket is created to maintain a connection with our web servers. 
But before creating the web socket, we must identify the user and check if he has permissions to participate in the conference. The user identification is done by retrieving the session id from the cookie provided by the user-agent (web browser) through the \ac{HTTP} headers.
%RP não é bem o user que fornece o cookie. é o user-agent (browser). Convém ser preciso.
%HR done
The web application server retrieves all the information needed from the database in order to check if the user has permissions to join that conference room. It is important to save the user identification before the \emph{WebSocket} connection is created because, after the handshake is performed by the \emph{WebSocket} protocol\cite{rfc6455}, the \ac{HTTP} context is lost.
%RP socket -> websocket?

When the connection is established between the application server and the client, a \emph{PeerConnection} is created on the client and immediately after, an \ac{WebRTC} endpoint is created on the server, specifying a possible set of \ac{ICE} servers to connect.
%RP Usa Application server e não apenas server. Há muitos servers.

At this stage, the web application's user is asked if he wants to share its camera and microphone, share screen or just receive streams from the server. If the user decides to perform a screen share, \emph{adapter.js}\footnote{\url{https://github.com/Temasys/AdapterJS}(accessed March 15, 2016).} may ask to install a plug-in if the browser does not support screen sharing \footnote{\url{http://iswebrtcreadyyet.com/} (Accessed May 11, 2016)}.
%RP web app user? OU apenas client user?
%RP foot note com exemplos de browsers que aceitem screen sharing?
%HR done
If the user decides to share either from camera or screen, \emph{getUserMedia} is called with the correspondent constraints in order to obtain a local stream. We use the constraints presented on Listing \ref{lst:constraints}. 

\begin{minipage}{\linewidth}
\begin{lstlisting}[caption={Media constraints},label={lst:constraints},language=JavaScript]
var screenShareConstraints = {	
	"video": {
		"mediaSource": "window" || "screen"
	}, 
	"audio": false
};
var cameraMicrophoneConstraints = {
	"audio":true, 
	"video":true 
};
var receiveOnlyConstraints = {
	"offerToReceiveAudio":true,
	"offerToReceiveVideo":true
};
\end{lstlisting}
\end{minipage}

If the user decides to share his camera or screen, a pop-up is raised in order to ask the user to give permission to share those resources. If the resources are shared successfully, the user agent creates an offer like Listing \ref{lst:sig01}, sets a \emph{local session description} to its \emph{PeerConnection} and sends it through the WebSocket to the Application Server.
%RP usas ``user'' duas vezes, uma para te referires ao utilizador e outra ao browser (user agent).
If the resources are not shared or the user specified to receive stream only, an \ac{SDP} offer is created specifying the constraints for receiving only video and audio.
%RP offer -> SDP offer?

\begin{minipage}{\linewidth}
\begin{lstlisting}[caption={Offer created by client},label={lst:sig01},language=json]
{
	"cmd":"offer",
	"data":{
		"type":"offer",
		"sdp":<sdp>	// omitted for brevity
	}
}
\end{lstlisting}
\end{minipage}

The \emph{local session description} contains the session identifier, codecs, containers, transport protocols and ports used per media type. The \emph{local session description} is useful to conclude if the client is receiving only, which means that \ac{KMS} does not need to mix, record nor analyze streams coming from the user's \ac{WebRTC} endpoint. 

The server receives and processes the offer and sets the \emph{remote session description} to its client associated \ac{WebRTC} endpoint. Then a \emph{local session description}, like the one presented in Listing \ref{lst:sig02}, is created on the server and sent back to the client. After that, the server tries to gather \ac{ICE} candidates.
%RP listing -> Listing

\begin{minipage}{\linewidth}
\begin{lstlisting}[caption={Answer created by KMS},label={lst:sig02},language=json]
{
	"cmd":"answer",
	"data":{
		"type":"answer",
		"sdp":<sdp>	// omitted for brevity
	}
}
\end{lstlisting}
\end{minipage}

The client receives the server answer, sets the \emph{remote session description} and gets the ice candidates from the \ac{ICE} server.
%RP tens ``ice'', ``\ac{ICE}''. Acho que também já vi ``ICE''. Convém ser sempre igual!

Subsequently, after a while both the server and client receive the \ac{ICE} candidates that allow the client to connect directly to \ac{KMS} and vice-versa. The candidates are received at the client which sets them to its \emph{PeerConnection}. The same is done on the server which receives the \ac{ICE} candidates like on Listing \ref{lst:sig03} from the client and propagates them to \ac{KMS}.
%RP A figura 4.1 precisa de vir antes deste texto todo, para ser mais fácil de perceber.
%RP também tens de explicar porque razão escolheste colocar o KMS numa rede privada. Se está numa rede privada, usas NAT ou um proxy (socket) para estabelecer a ligação?
%HR uso NAT, falei disso na arquitectura, serve para o cliente não aceder directamente ao servidor de streaming e realizar o seu proprio protocolo de sinalizacao (uso indevido por parte de atacantes)

\begin{minipage}{\linewidth}
\begin{lstlisting}[caption={ICE candidates sent by KMS and client},label={lst:sig03},language=json]
{
	"cmd":"iceCandidate",
	"data":{
		"sdpMLineIndex":0,
		"candidate":"candidate:15 1 TCP 843056127 146.193.224.20 48828 typ srflx raddr 192.168.1.105 rport 48828 tcptype passive",
		"sdpMid":"audio"
	},
}
\end{lstlisting}
\end{minipage}

An \ac{ICE} candidate contains an \ac{IP}, port, used transport protocol and an attribute named \emph{sdpMLineIndex} that is used for mapping to the \emph{remote session description} media type.
In addition, both intervenients test the connectivity of each \ac{ICE} candidate. When a connection is established, the user and server start to interchange stream data but other \ac{ICE} candidates may arrive with better connections. When that happens, the connection changes seamlessly. 

Having the media session established, the server starts to record any received stream and the client creates an \ac{URL} correspondent to the stream location.


%RP não falas nada sobre chats, etc! Não há signalling para isso?
%HR não, só video e audio, chat é websockets
\section{Stream Recording}
	In this section we describe our approaches for the stream recording implementation, namely recording on the client side and server side.

\subsection{Client side recording}
	In order to record used shared streams, our first approach consisted on recording video and audio streams into blobs of limited duration. Because each user was recording directly from their shared local resources, we achieved the best stream quality.

	After recording, each block was uploaded to our server through \ac{HTTP} and saved into the filesystem, the block metadata was created and inserted on our database. Each block metadata contained the file's location for the recorded block, the starting date, duration, user identification and group identification. When the file was completely saved the metadata was advertised to the remaining users. This metadata was simply used to refresh the user interface, the metadata was completely discard after that because an huge amount of small blocks metadata would use more and more memory as time went by. 

	The process to play a video was quite simple, the user specified which user and date was intended to play, the server calculated the intersection between the requested date and the block bounds and returned the file to the client.

	Altough this idea was fairly simple, we couldn't achieve seamless sequential block switching. Downloading the video file always toke a noticeable amount of time. To solve this problem while we were playing a block, the next one was downloading in parallel so when the current block finished playing we would have an available block to play. 

	Block switching became more acceptable, but switching the url always produced a flash. We solved this problem by having two layers and set the url to the back layer, the front layer freezed with the last frame, then we changed the back layer to front. 

	After we implemented our recording solution using this approach, we tested localy and remotely. For remote connections we observed a fairly high bandwidth usage mainly because blocks were both sent and received at maximum quality. 

\subsection{Server side recording to filesystem}

	Before recording any type of stream we had to analyse the user media offer in order to check if a video as realy being received by \ac{KMS}, otherwise if we would not verify the user's offer, the recording video would be black.

	The streaming content received on \ac{KMS} was already compressed due to \ac{WebRTC}'s exchanged \ac{QoS} metrics data. As a direct consequence our recorder solution used on most cases less disk space per block, but would never use more storage than client side recording. \ac{KMS} allows recording files using \emph{webm} and \emph{mp4} containers.

	With server side recording, the user would maintain always the same stream url even if it is playing realtime video or reproducing recorded video. When a user desires to play recorded video, a websocket message is sent specifying the time and the intended user id, the group identification is not sent because it is already associated to the websocket. The server performs the same calculations in order to find a block that intersects the requested time, plays it and when finished the next part is automatically played wihtout the user intervention.

	We observed differences in image quality when switching parts, that was even noticeable if we set a short block duration. We also noticed a small gap on audio when switching blocks but it was acceptable and speech recognition was not very affected. 

	Back when we were implementing our solution, \ac{KMS} had not support for seeking videos, which meant that blocks would would always start playing from their beggining. This lead to a theoric playing time error that can be at most half the duration of a block. In practice the maximum playing time error coincides with the block duration because we perform the intersection between the requested date and the block but we could decrease the error by half if we calculate the intersection with the requested time plus half the duration. 

	\begin{figure}
			\centering

	\begin{tikzpicture}[y=1cm, x=1cm, thick, font=\footnotesize]    

		\tikzset{
		   brace_top/.style={
		     decoration={brace},
		     decorate
		   },
		   brace_bottom/.style={
		     decoration={brace, mirror},
		     decorate
		   }
		}

		% time line hour
		\draw[line width=1.2pt, ->, >=latex'](0,-3.0) -- coordinate (x axis) (9,-3.0) node[right] {time}; 
		\foreach \x in {1,2,3,4,5} \draw (\x*1.5,-2.9) -- (\x*1.5,-3.1) node[below] {};

		% top brace
		\draw [brace_top] (1*1.5+0.03,-2.7) -- node [above, pos=0.5] {$ block_{n}$} 	(3*1.5-0.03,-2.7);
		\draw [brace_top] (3*1.5+0.03,-2.7) -- node [above, pos=0.5] {$ block_{n+1}$} 	(5*1.5-0.03,-2.7);

	    \draw  node[fill,circle,scale=0.6]at (2*1.5,-3.0)  {};%

		% low brace period
	
		\draw [brace_bottom] (1*1.5+0.03,-3.3) -- node [below, pos=0.5] {$e_{n}=\frac{d}{2}$} 		(2*1.5-0.03,-3.3);
		\draw [brace_bottom] (2*1.5+0.03,-3.3) -- node [below, pos=0.5] {$e_{n+1}=\frac{d}{2}$} 	(3*1.5-0.03,-3.3);
		\draw [brace_bottom] (3*1.5+0.03,-3.3) -- node [below, pos=0.5] {$d$} 						(5*1.5-0.03,-3.3);

	\end{tikzpicture}
		\caption{Theoretic maximum playing error $(e)$}
	\end{figure}

	\begin{figure}
			\centering
	\begin{tikzpicture}[y=1cm, x=1cm, thick, font=\footnotesize]    

		\tikzset{
		   brace_top/.style={
		     decoration={brace},
		     decorate
		   },
		   brace_bottom/.style={
		     decoration={brace, mirror},
		     decorate
		   }
		}

		% time line hour
		\draw[line width=1.2pt, ->, >=latex'](0,-3.0) -- coordinate (x axis) (9,-3.0) node[right] {time}; 
		\foreach \x in {1,2,3,4,5} \draw (\x*1.5,-2.9) -- (\x*1.5,-3.1) node[below] {};

		% top brace
		\draw [brace_top] (1*1.5+0.03,-2.7) -- node [above, pos=0.5] {$ block_{n}$} 	(3*1.5-0.03,-2.7);
		\draw [brace_top] (3*1.5+0.03,-2.7) -- node [above, pos=0.5] {$ block_{n+1}$} 	(5*1.5-0.03,-2.7);

	    \draw  node[fill,circle,scale=0.6]at (3*1.5,-3.0)  {};%

		% low brace period
	
		\draw [brace_bottom] (1*1.5+0.03,-3.3) -- node [below, pos=0.5] {$e_{n}=d$} 	(3*1.5-0.03,-3.3);
		\draw [brace_bottom] (3*1.5+0.03,-3.3) -- node [below, pos=0.5] {$d$} 		(5*1.5-0.03,-3.3);

	\end{tikzpicture}
		\caption{Practical maximum playing error $(e)$}
	\end{figure}



	For playing video with an higher velocity we used \emph{ffmpeg}\footnote{\url{https://www.ffmpeg.org/}(accessed: 17 March 2016)} to convert the block into a new video with the desired velocity and seek time. Because the media duration is known, when the video started to being convert the headers located at the begining of the file were already written and that made it possible to stream while converting.

	%{\color{red} [TALK ABOUT COMPLEXITY OF DECODE VS DECODE+MANIPULATE+ENCODE]}

	Although we implemented a solution that worked, we imediately noticed that \emph{ffmpeg} would take some time to initialize and that lead to pauses between switching parts.

	Later the \emph{Kurento} team released a version with support for seeking videos but we had to suspend the implementation of the fast forwarding feature as currently \ac{KMS} is not supporting that.

	We could implement fast forwarding without real time conversion by creating multiple versions of the same video with different velocities after the recording of a block. When a user needed to play he would also need to specify one of the available velocities. We did not followed this approach has it would require a bigger disk space usage. 

\subsection{Server side recording to database}

	One of our concerns during the development of our solution was the storage scalability. Saving files directly into filesystem would require an extra effort to distribute and replicate files among servers. For that purpose \emph{Kurento} team developed \emph{Kurento Repository}\footnote{\url{http://doc-kurento-repository.readthedocs.org} (accessed on 17 March 2016)} which is based on \emph{MongoDB}.

	One of the features that \emph{Kurento Repository} provides is the ability to play directly from the database without having to download the entire file to \ac{KMS}. The same is true for recording, but because the file headers are in the beggining and the file is written until is stops, the headers don't contain the necessary information for seeking the file.

	Although we gain with scalability with this approach, we lose access over the file for changing it to fit our needs, namely for using \emph{ffmpeg} or other video manipulation tool.

	Although we did not implemented recorded file seeking, that could be achieved by waiting for full file recording and then proceed to database insertion with the correct headers. Another approach would be the specification of the file duration before recording so the correct file headers could be written a priori. Both approaches were not possible to implement using just the \emph{Kurento} clients, we would need change the source code of \emph{Kurento} in order to add those new features.






\section{Hyper Content}
	In this section we describe the algorithm we created to show synchronized interactive content to clients.



	\subsection{Content creation}

	In order to create content the user has the option to write simple movie captions without writing any code, otherwise, as mentioned before, it can write \ac{HTML}, \ac{CSS} and \emph{JavaScript}. The definition of the content's starting and ending time by the user it's not an easy task. Defining the content's time to appear in real-time would require a previous user plan. Otherwise the user could make a speech and add the content for later reproduction.

	Figure \ref{fig:creation} shows the user interface for creating interactive hyper content manually.

	\begin{figure}[H]
		\centering
		\includegraphics[width=0.6\textwidth]{figures/edition.png}
		\caption{Manual content creation}
		\label{fig:creation}
	\end{figure}


	Listing \ref{lst:createcontent} shows the structure of the content created by one user.

\begin{minipage}{\linewidth}
\begin{lstlisting}[caption={Exampe of content created by one user},label={lst:createcontent},language=json]
{
	"cmd":"createContent",
	"start":"2016-03-29T03:56:40.000Z",
	"end":"2016-03-29T03:56:41.000Z",
	"content":"<h1>Hello</h1>"
}
\end{lstlisting}
\end{minipage}
	In order to help the content creator to show and synchronize its content in real-time we allow the user to encode its content into \emph{QR codes} and show it to the camera in real time.

	\ac{KMS} lets registering event handlers for \emph{QR code} detection. The component that detects \emph{QR codes} on \ac{KMS} is called periodically and fires the handlers with the decoded content. This mechanism does not detect if the \emph{QR code} enters or leaves the screen. We had to implement our own mechanism for detecting those events.

	Each user session in the server maintains a map with the contents that are present on the screen. In order to apply our algorithm more efficiently we calculate the content's hash through the \emph{md5} method.

	If the hash is not present in the map it means the \emph{QR code} was entering the screen, we add that hash to the map and associate the current time to it, all the users are notified to watch that content in real-time. By doing an analogy to the \emph{Nyquist} theorem we know that if the same \emph{QR code} is not detected after a time bigger then two periods we can conclude that the \emph{QR code} leaved the screen and we add the correspondent content into our database. Listing \ref{lst:pseudo_qrcode} shows the pseudo code for \emph{QR code} leaving detection.

\begin{minipage}[!htb]{\linewidth}
\begin{lstlisting}[caption={Pseudo code for QR code leaving detection},label={lst:pseudo_qrcode},language=JavaScript]
var ongoingCodesMap = {};
function onCodeFound(content) {
	var startingTime = getCurrentTime() - getUserOffset();
	var hash = md5(content);
	if(ongoingCodesMap[hash] == null){
		sendQRCodeToEditingUser(hash,content);
	}
	ongoingCodesMap[hash] = startingTime;
	waitTwoSeconds(); // onCodeFound() may be called from other threads
	var newestTime = ongoingCodesMap[hash]; 
	if(newestTime == startingTime){
		var endingTime = getCurrentTime() - getUserOffset();
		ongoingCodesMap[hash] = null; 
		sendQRCodeToEditingUser(hash,null); // remove from UI
		insertContentIntoDatabase(startingTime,endingTime,content);
		sendContentToUsers(startingTime,endingTime,content);
	}
}
\end{lstlisting}
\end{minipage}


	Our main content synchronization mechanism is time based but with some programming knowledge it is possible to insert \emph{JavaScript} code that fires events on user interaction. For example, after a teacher's lecture, it is possible to show a quiz to the users in order to understand what they learned and then submit the data to the server for further analysis.


	\subsection{Synchronization}

	In our solution content is represented as simple text that can contain \ac{HTML}, \ac{CSS} or even \emph{JavaScript}, this content is displayed on defined intervals of time.

	Multiple contents can be displayed at the same time, in order to achieve that, we define layers above the video with the same size. Each layer is associated to the content.

	We have taken into account that the amount of content tends to grow with time and the user should only have access to a subset of content instead all of them, which would be very inefficient.

	The content to display for each user depends on the user's position on the time-line, this position is given by an offset between the current time and the navigated time. If the user is watching the content in real-time this offset is constant so there is no need to synchronize this value.

	Each content is divided into two components, the \emph{start} and the \emph{end} which are sent to users. The \emph{start} component contains a time stamp, the content identification number and the content itself in form of text. The \emph{end} component only contains the time stamp and the content identification number. 

	The content to return is given by the union of two content subsets:
	\begin{itemize}
		\item The intersection of the content interval and user's time.
		\item A subset of contents which starting time is immediately following the user's time.
	\end{itemize}

	The second subset is used for predicting which content the user will watch and avoid requests during its events.

	The content description messages that are sent to each user contains the constant itself on the form of a list of events and an attribute that specifies if the server has more content to return.

	Listing \ref{lst:sentcontent} shows the structure of the content sent to users.

\begin{minipage}{\linewidth}
\begin{lstlisting}[caption={Exampe of content sent to users},label={lst:sentcontent},language=json]
{
	"cmd":"content",
	"data":[{
			"id":"56f9fd1aa986c615fab43d69",
			"time":"2016-03-29T03:56:40.000Z",
			"type":"start",
			"content":"<h1>Hello</h1>"
		},{
			"id":"56f9fd1aa986c615fab43d69",
			"time":"2016-03-29T03:56:41.000Z",
			"type":"end"
		}
	],
	"more":false
}
\end{lstlisting}
\end{minipage}

	When a user enters the conference room, immediately after the \emph{WebSocket} creation the server sends him the current content.

	The user receives the content, sorts all components by time and creates a set of events. All the events before the user's time are rendered and removed from the set of events. A timer is scheduled for the first component on the event set and the process repeats while the set is not empty.

	If the set is empty, there two options, if the server contained more content a new request for content is made and the process starts from the beginning, namely the server sends the correspondent content again. If the server has no more content the process is stopped until it sends more.

	The users will receive their contents from the server on five different situations:

	\begin{itemize}
		\item Conference room entrance (advertised by server).
		\item Set of events empty and server has more content (client requested).
		\item Content is created (advertised by server).
		\item Content is removed (advertised by server).
		\item User navigates to different point in time (client requested).
	\end{itemize}



	Listing \ref{lst:pseudo_render} presents the pseudo code for our content scheduler.

\begin{minipage}[!htb]{\linewidth}
\begin{lstlisting}[caption={Pseudo code for hyper content scheduler},label={lst:pseudo_render},language=JavaScript]
function scheduleContent() {
    var navigatedTime = getCurrentTime() - getUserOffset();
    while ( hasLocalEvents() ) {
        if ( firstEventIsOlderThanNavigatedTime() ) {
            if ( firstEventIsStart() ) {
            	addHtmlLayer(event);
            } else {
            	removeHtmlLayer(event);
            }
            removeFirstEvent();
        } else {
            break;
        }
    }
    if ( hasNoLocalEvents() ) {
    	return;	// nothing to do
    } 
    if ( hasNoStartEvents() && serverHasMoreContent() ) { 
       	requestContentFromServer(navigatedTime);
    } else {
        waitUntilNextEvent();
        scheduleContent();
    }
}
\end{lstlisting}
\end{minipage}

	\subsection{Security Concerns}

	Our solution is flexible on what kind of interactions are possible to the users in real time but allowing users to write \emph{JavaScript} that is executed on the other users browser would attackers to misuse their resources and access to critical information.

	We could solve this problem easily by escaping any \emph{script} tag present on the content, but we would sacrifice the kind of interactions that are possible. 

	By not allowing \emph{JavaScript} we would need to implement all actions a priori and fire them when a type of message is received. We decided to ignore the security vulnerabilities that are exposed by evaluating \emph{JavaScript} because in order to offer the same interactions we would have to do an exhaustive functional requirements gathering.

	Another way to solve this problem, which is not within the goals of this thesis, is to analyze the \emph{JavaScript} code and detect if it is malicious.

	\subsection{Time Manipulation}

	We have used \emph{vis.js} to display out time-line. This library was created for content navigation through time, but it was not designed to be always moving automatically, we have created a background timer that performs the animation of moving the window of time bounds and user navigated time marker.

	Figure \ref{fig:timeline} shows the the graphic appearance of our interactive time-line, in order to navigate through time the user must drag and drop the time-line horizontally, when the user drops the time-line an event handler is called with the new user's time offset which will be used to send a message to the server in order to choose the correspondent content to display. 

	\begin{figure}[!htb]
		\centering
		\includegraphics[width=\textwidth]{figures/timeline.png}
		\caption{Interactive time-line}
		\label{fig:timeline}
	\end{figure}

	We have noticed the server time being different from client time and this created graphic inconsistencies such as showing existing recorded blocks of movie in the future. To solve this problem the server sends its time to client immediately after the \emph{webSocket} creation, we synchronize the time-line with the server and although it may exist a small error due to network transmission time the graphical error is negligible. 


	\subsection{Time annotations creation}

	Time annotations are simpler way save points in time and share them with other users. When a user enters in a conference room the server sends all annotations so they can be displayed directly into the user time-line. Each time annotation is created all users are advertised so they can update their interfaces.

	Figure \ref{fig:annotation} shows the creation and placement of annotations on the time-line, to save the annotation the user must click on the floppy icon in order to save it and advertise the conference participants. 

	\begin{figure}[!htb]
		\centering
		\includegraphics[width=0.7\textwidth]{figures/annotation.png}
		\caption{Creating a time annotation}
		\label{fig:annotation}
	\end{figure}

	Besides the ability to create tags it is also possible to create time hyper-links, see figure \ref{fig:timelink}, that can be sent to other users externally so they can navigate directly to the content when entering in the conference room.


	\begin{figure}[!htb]
		\centering
		\includegraphics[width=0.8\textwidth]{figures/timelink.png}
		\caption{Time link creation}
		\label{fig:timelink}
	\end{figure}


	\subsection{Content Search}

	Users can easily search for annotations and contents and travel to their correspondent times. In the case of hyper content after handling the result from database we ignore \ac{HTML} tags, extract the text with \emph{Jsoup}\footnote{\url{http://jsoup.org/} (Accessed 21 March 2016)} and apply the query again. We extract the text from \ac{HTML} because accidentally searching for text contained in \ac{HTML} tags would lead to incorrect results.

	Figure \ref{fig:search} shows how the search results are displayed to the user. Each result entry contains an icon specifying the type of result (hyper content, time annotation, ...).




\begin{figure}[!htb]
\centering
\begin{minipage}[b]{0.55\linewidth}
\centering

		\includegraphics[width=\textwidth]{figures/search.png}
	      a) Result list
\label{fig:minipage1}
\end{minipage}
\quad
\begin{minipage}[b]{0.40\linewidth}
		\centering

		\includegraphics[width=\textwidth]{figures/search2.png}
	       b) Selected result
\label{fig:minipage2}
\end{minipage}

		\caption{Example of search results}
		\label{fig:search}
\end{figure}

	\subsection{Switching views}

		We provide a way for users to select the composite view of the conference room or the video shared by a particular user device as it can be seen on figure \ref{fig:devices}. 

		In order to maintain a list of available user devices that can be played at the navigated time if a user plays or ends playing a block of recorded video it obtains at the same time a list of all devices available for each user at that time.

		On the other hand, if a user changes to real time, the list of devices is extracted from the set of \emph{webSockets} that are associated to the respective conference room. This list of devices is also sent to the users that are also in real time mode whenever a new user enters or leaves the conference room.

	\begin{figure}[!htb]
		\centering
		\includegraphics[width=0.8\textwidth]{figures/devices.png}
		\caption{Multiple devices per user}
		\label{fig:devices}
	\end{figure}

		We also provide an automatic mechanism for switching the view to the current speaker. With respect to sound analysis, the sound samples are analyzed on the client side through the web audio \ac{API}.

		Our speech detector is straightforward, we could perform a spectral analysis in order to understand if the analyzed sound contains frequencies in the range of the human voice but instead we just capture sound samples in real time and calculate the maximum sound amplitude. If a sound sample has an amplitude bigger than a factor of the maximum amplitude (we have used an empirical value of 10\%) we say that the user is speaking and therefore we send a message to the server if the speaking state has changed. Subsequently, the server receives the user speaking state and sends it to the other users so they can request a different view.
\section{Solution deployment}
	\label{section:deployment}

	In this section we describe the hardware and software we have used to deploy our solution. 

	\subsection{Hardware}

	The hardware was gently provided by INESC-ID\footnote{\url{http://www.inesc-id.pt/} (Accessed March 26, 2016)}, the specifications of the server that we used are specified on table \ref{table:hwspecs}.

	\begin{table}[!htb]
\centering
\caption{Hardware specifications}
\label{table:hwspecs}
\begin{tabular}{|l|l|}
\hline
Server & Supermicro SuperServer 6027R-72RF\footnotemark \\ \hline
CPU & 2 x Intel Xeon E5-2640V2, LGA 2011, 2.0GHz, 8C/16T \\ \hline
RAM & 8 x DDR3 REG16G-1600DDR3, 16GB, DDR3-1600, Registered ECC, memory \\ \hline
Network cards & 2 x Intel Corporation I350 Gigabit Network Connection (rev 01)  \\ \hline
Disks & \begin{tabular}[c]{@{}l@{}}2x SAMSUNG SSD 840 PRO 256GB SATA III (Drive 0 - RAID 1 - 237.486 GB)\\
4x WESTERN DIGITAL 3TB SATA III 64MB RED (Drive 1 - RAID 5 - 8.185TB)\end{tabular} \\ \hline
\end{tabular}
\end{table}

 \footnotetext{\url{http://www.supermicro.com/products/system/2U/6027/SYS-6027R-72RF.cfm} (Acessed March 26, 2016)}

\subsection{Operating System}
	
On table \ref{table:osspecs} we present an overview of the operating system configurations of the machine we used for deployment.

	\begin{table}[!htb]
\centering
\caption{Operating system specifications}
\label{table:osspecs}
\begin{tabular}{|l|l|}
\hline
Operating System & Linux version 3.16.0-4-amd64 \\ \hline
Distribution & Debian GNU/Linux 8 (jessie)\footnotemark \\ \hline
Swap & 16GB (Drive 0)\\ \hline
Root & 40GB (Drive0)\\ \hline
EFI & 200MB (Drive 0)\\ \hline
Bcache & Remaining space (Drive 0 \& Drive 1) \\ \hline
\end{tabular}
\end{table}

\footnotetext{\url{https://www.debian.org/} (Accessed March 26, 2016)}

\subsection{Software}
	INESC-ID provided us a restricted linux account without administration permissions, which we prevented us the installation of our solution directly on the machine because we would need administrative priviledges to install all the software that our solution depends on. Although that limitation, they provided access to Docker\footnote{\url{https://www.docker.com/} (Accessed March 27, 2016)} which run on host operating system as an isolated process in userspace.

	Inside a Docker container we have the administrative priviledges to install all the software dependencies.

	We could use one docker image for each component but in order to reduce the network usage we prefered to install all the components within the same image. At the same we provide an all-in-one easy to deploy solution as all the \ac{IP}s are local.

	We present, on table \ref{table:softspecs}, the software we have installed inside our docker container including the versions that are in use.

	\begin{table}[!htb]
\centering
\caption{Installed Software}
\label{table:softspecs}
\begin{tabular}{|l|l|}
\hline
\multicolumn{1}{|c|}{\textbf{Name}} & \multicolumn{1}{c|}{\textbf{Version}}         \\ \hline
Ubuntu Server\footnotemark & 14.04 LTS   \\ \hline
Oracle Java & 1.8.0\_77   \\ \hline
Play Framework & 2.5.0  \\ \hline
MongoDB & 3.0.10   \\ \hline
Kurento Media Server & 6.4.0 \\ \hline
Kurento Repository & 6.3.1 \\ \hline
Python & 2.7.6 \\ \hline
\end{tabular}
\end{table}
 \footnotetext{\url{http://www.ubuntu.com/} (Accessed March 27, 2016)}




\section{Chapter Summary}
\label{implementation:summary}

In this chapter we have described how we implemented the various components present on our architecture, the challenges that we have faced and the solutions we have found.

We started by defining the database model, which influenced directly our system's behavior and the extensibility to new functionalities.

Then, we underlined the requirements of our signaling protocol and, as a consequence, we have described, in detail, the protocol itself.

With the signaling protocol implemented, we observed the first outcomes of using \ac{WebRTC}, the basic functionality we have implemented was an echo of the video and audio sent by users. However we have implemented other features such as switching to other user streams, recording and mixing multiple streams into a single one.

In respect to hyper-content, we have described how to create and search for content either being superimposed content to video and time annotations. In relation to displaying content to users we have also defined how we have synchronized the content to show and the security concerns of our choices.

Then, we have described how we have implemented our chat and collaborative editor.


