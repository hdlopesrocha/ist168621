\section{Chapter Summary}
\label{related:summary}

In order to implement an hyper-linked communication solution, several design decisions had to be made. The limitations imposed by the \emph{Internet's} structure, its protocols and a browser's capabilities are key factors to consider when implementing a solution that allows bi-directional communications, interactive media and collaboration environment.

Due to the use of \ac{NAT}, bi-directional communications between clients have different needs from request-response based communications between clients and servers.
This lead to the appearance of mechanisms such as \ac{STUN}, \ac{TURN} and \ac{ICE}, in order to bypass the limits imposed by \ac{NAT}.

\ac{WebRTC} introduced an \ac{API} for video, audio and data communications through web browsers without using plug-ins.
However, \ac{WebRTC} by itself does not define how users get to know each other nor how information flows between users. For this reason, we have studied the multiple ways we could implement this \emph{get-to-know} mechanism which is known as signaling protocol.

With the communications establishment issue solved, we had to discuss the different types of media and what can be done with each kind in order to increase the value of communications among users. In this context, we have studied solutions and libraries that allow us to implement our prototype with time manipulation features, collaborative text edition, record and playback interactive video. 
%RP falas de mais coisas: manipulação de tempo, gravação, edição colaborativa
