\chapter*{Abstract}

% DO NOT CHANGE THIS - Add entry in the table of contents as chapter
\addcontentsline{toc}{chapter}{Abstract}

The Hyper-linked communications concept applies much of the hypermedia concepts, widely used on Web content, to communication and collaboration services. This paradigm allows to synchronize, structure and navigate communication content, providing a way to integrate social media content and collaborative tools into voice and video calls.
Voice and image together can express emotions and expose creativity like no other medium can. With hypermedia concepts, we can add more value to voice and video communications.

\ac{WebRTC} technology allows real-time communications between web browsers without the need to install and use additional applications or plug-ins. The nature of web browser applications already follows the hypermedia concept, which makes \ac{WebRTC} the ideal technology to apply the hyper-linked communications concepts.
{\color{blue}The web browser platform provides an abstraction layer that makes it possible to create applications that run independently from the operating system.}
The native support for \ac{WebRTC} in operating systems extends its usage to outside the web browser, allowing for the exploration of functionalities for which web browsers provide poor support, such as video recording and massive information storage.

%\sout{Our goal in this project is to develop an application that leverages the Hyper-linked communications concept to enable collaborative real-time audio and video communication, enriched with other media types, that can be accessed live or accessed later.}
%This application will target the web browser platform, resorting to \ac{WebRTC}.


{\color{red}Our goal in this project is to develop an application targeted to the web platform, resorting to \ac{WebRTC}, that leverages the hyper-linked communications by providing a video conference environment enriched with interactive and non-interactive discrete media types such as images, subtitles, forms and all types of content that can be added using \emph{HTML5}, \emph{CSS3} and \emph{JavaScript} including continuous media types such as video, music and animations.}

{\color{red}One of the key features of this project is the ability to navigate in time in order to reproduce the conversation again or introduce hyper-content to it such as time annotations, interactive lists of topics and subtitles. In this context we also provide a simpler method for creating and synchronizing hyper-content using \emph{QR codes}.}

{\color{red}In addition to this conference environment, which provides different functionalities than traditional conference environments such as \emph{Skype} and \emph{Google Hangouts}, we also enable a collaborative text editor and a chat that supports sending time hyper-links and files to conference participants.}

{\color{red}Furthermore, another relevant feature is the possibility to compose multiple video streams into a single one, which enables adding more users to conference rooms without impacting on clients performance. Users can change to individual streams on demand or automatically to the talking users.}


%RP acho que estás a vender muito mal o teu trabalho. O abstract é a primeira caisa que se lê e em função disso decidimos se lemos mais ou não. Tens aqui a oportunidade de chamar a atenção para algumas das funções inovadoras. O texto está muit genérico. podes falar em coisas concretas: video conferência, chat, edição colaborativo de documento, partilha de ficheiros, possibilidade de criar legendas, tags, etc. Possibilidade de navegar, voltar atrás, etc como no IPTV. Gestão de amigos, gerir salas, etc, etc.

{\color{blue}
In this document, we present the current State Of The Art in hyper-linked communications and related technologies, propose and implement an architecture for an hyper-linked communication application based on \ac{WebRTC} and provide the results for its implementation and evaluation.
}

%RP em vez de ``provide the results for its implementation and evaluation'', diz algo como: ``this work was evaluated using a panel of 20 users, who reported that they liked to use it and thought it to be extremelly inovative ... and if given the change, would use it ...''

\vspace{1cm}

% TODO 4 to 6 keywords;
\textbf{\Large Keywords:} WebRTC, asynchronous, communications, collaboration


\cleardoublepage
