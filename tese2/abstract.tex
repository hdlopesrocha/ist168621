\chapter*{Abstract}

% DO NOT CHANGE THIS - Add entry in the table of contents as chapter
\addcontentsline{toc}{chapter}{Abstract}

The Hyper-linked communications concept applies much of the hypermedia concepts, widely used on Web content, to communication and collaboration services. This paradigm allows to synchronize, structure and navigate communication content, providing a way to integrate social media content and collaborative tools into voice and video calls.
Voice and image together can express emotions and expose creativity like no other medium can. With hypermedia concepts, we can add more value to voice and video communications.

\ac{WebRTC} technology allows real-time communications between web browsers without the need to install and use additional applications or plug-ins. The nature of web browser applications already follow the hypermedia concept, which makes \ac{WebRTC} the ideal technology to apply the hyper-linked communications concepts.
{\color{blue}The web browser platform provides an abstraction layer that makes possible to create applications that run indenpendently from the operating system.}
The native support of \ac{WebRTC} in operating systems extends its usage to outside the web browser, allowing to explore functionalities that web browsers are poor to support such as video recording and massive information storage.

Our goal in this project is to develop an application that leverages the Hyper-linked communications concept to enable collaborative real-time audio and video communication, enriched with other media types, that can be accessed live or accessed later.
This application will target the web browser platform, resorting to \ac{WebRTC}.

{\color{blue}
In this document, we present the current State Of The Art in hyper-linked communications and related technologies, propose and implement an architecture for an hyper-linked communication application based on \ac{WebRTC} and provide the results for its implementation and evaluation.
}

\vspace{1cm}

% TODO 4 to 6 keywords;
\textbf{\Large Keywords:} WebRTC, asynchronous, communications, collaboration


\cleardoublepage
