\section{Stream Recording}
In this section we describe our approaches for the stream recording implementation, namely recording on the client side and server side.
%RP Innitially we experimented with local recording. Even thought it got it to work, it consumed to much bandwith. We then tried ....  In this section, we describe the several techniques evaluated.

\subsection{Client side recording}
In order to record used shared streams, our first approach consisted in locally recording video and audio streams into blobs of limited duration. Because each user was recording directly from their shared local resources, we achieved the best stream quality.
%RP used -> user?

After recording, each block was uploaded to our server through \ac{HTTP}, saved into the server's file system, the block meta-data was created and inserted on our database. Each block meta-data contained the file's location for the recorded block, the starting date, duration, user identification and group identification.
%RP referenciar o modelo na secção 3.xx?
When the file was completely saved, the meta-data was advertised to the remaining users. This meta-data was simply used to refresh the user interface and was completely discarded after that, because an huge amount of small blocks' meta-data would use more and more memory as time went by. 

	The process to play a video was quite simple. The user specified which user and date he intended to play, the server calculated the intersection between the requested date and the block bounds and returned the file to the client.
	Although this idea was fairly simple, we couldn't achieve seamless sequential block switching. Downloading the video file always took a noticeable amount of time. To solve this problem, while we were playing a block, the next one was downloading in parallel, so when the current block finished playing we would have an available block to play. 

        With this technique, block switching became more acceptable, but switching the \ac{URL} always produced a flash. We solved this problem by having two layers and changing the \ac{URL} on the back layer. When the front layer froze on the last frame, we changed the back layer to the front. 

	After we implemented this recording solution using this approach, we tested locally (client and server on the same machine) and remotely. For remote connections we observed a fairly high bandwidth usage, mainly because blocks were both sent and received at maximum quality. 
        %RP We deemed this high bandwidth usage to be a significant bottleneck on our application and decided to explore other solutions.
        
\subsection{Server side recording to file system}
%RP começar por explicar razão para usar KMS: permite gravar e juntar várias streams numa só, o que torna a aplicação muito mais escalável.

Before recording any type of stream, we had to analyze the user media offer in order to check if a video was really being received by \ac{KMS}, otherwise if we did not verify the user's offer, the recording video would be black.
%RP Entras logo em grandes pormenores. Falta aqui uma introdução.

The streaming content received on \ac{KMS} was already compressed due to \ac{WebRTC}'s exchanged quality of service metrics data. As a direct consequence, our recorder solution used on most cases less disk space per block, but would never use more storage than client side recording. \ac{KMS} allows recording files using \emph{webm} and \emph{mp4} containers.
%RP escolheste qual?

	With server side recording, the user would maintain always the same stream \ac{URL} even if it is playing real-time video or reproducing recorded video. It is \ac{KMS} that sends different content through that stream. When a user desires to play recorded video, a \emph{webSocket} message is sent specifying the time and the intended user id. The group identification is not sent because it is already associated to the \emph{webSocket}. The server performs the same calculations in order to find a block that intersects the requested time, plays it and when finished, the next part is automatically played without the user intervention.

	We still observed differences in image quality when switching parts. That was even noticeable if we set a short block duration. We also noticed a small gap on audio when switching blocks but it was acceptable and speech understanding was not very affected. 

	When we started implementing our solution, \ac{KMS} had no support for seeking videos, which meant that blocks would would always start playing from their beginning. This lead to a theoretic playing time error that can be at most half the duration of a block. In practice, the maximum playing time error coincides with the block duration because we perform the intersection between the requested date and the block. But we could decrease the error by half if we calculate the intersection with the requested time plus half the duration.
        %RP já tinhamos falado sobre isto. O erro é no máximo o tamanho do bloco pois queres garantir que a pessoa ve sempre o que solicita. Assim começas sempre no bloco que fica antes e nunca no que fica depois.

        %RP Figura tem de ser apresentada no texto e explicada. O que significam as letras?
        
	\begin{figure}
			\centering

	\begin{tikzpicture}[y=1cm, x=1cm, thick, font=\footnotesize]    

		\tikzset{
		   brace_top/.style={
		     decoration={brace},
		     decorate
		   },
		   brace_bottom/.style={
		     decoration={brace, mirror},
		     decorate
		   }
		}

		% time line hour
		\draw[line width=1.2pt, ->, >=latex'](0,-3.0) -- coordinate (x axis) (9,-3.0) node[right] {time}; 
		\foreach \x in {1,2,3,4,5} \draw (\x*1.5,-2.9) -- (\x*1.5,-3.1) node[below] {};

		% top brace
		\draw [brace_top] (1*1.5+0.03,-2.7) -- node [above, pos=0.5] {$ block_{n}$} 	(3*1.5-0.03,-2.7);
		\draw [brace_top] (3*1.5+0.03,-2.7) -- node [above, pos=0.5] {$ block_{n+1}$} 	(5*1.5-0.03,-2.7);

	    \draw  node[fill,circle,scale=0.6]at (2*1.5,-3.0)  {};%

		% low brace period
	
		\draw [brace_bottom] (1*1.5+0.03,-3.3) -- node [below, pos=0.5] {$e_{n}=\frac{d}{2}$} 		(2*1.5-0.03,-3.3);
		\draw [brace_bottom] (2*1.5+0.03,-3.3) -- node [below, pos=0.5] {$e_{n+1}=\frac{d}{2}$} 	(3*1.5-0.03,-3.3);
		\draw [brace_bottom] (3*1.5+0.03,-3.3) -- node [below, pos=0.5] {$d$} 						(5*1.5-0.03,-3.3);

	\end{tikzpicture}
		\caption{Theoretic maximum playing error $(e)$}
	\end{figure}

	\begin{figure}
			\centering
	\begin{tikzpicture}[y=1cm, x=1cm, thick, font=\footnotesize]    

		\tikzset{
		   brace_top/.style={
		     decoration={brace},
		     decorate
		   },
		   brace_bottom/.style={
		     decoration={brace, mirror},
		     decorate
		   }
		}

		% time line hour
		\draw[line width=1.2pt, ->, >=latex'](0,-3.0) -- coordinate (x axis) (9,-3.0) node[right] {time}; 
		\foreach \x in {1,2,3,4,5} \draw (\x*1.5,-2.9) -- (\x*1.5,-3.1) node[below] {};

		% top brace
		\draw [brace_top] (1*1.5+0.03,-2.7) -- node [above, pos=0.5] {$ block_{n}$} 	(3*1.5-0.03,-2.7);
		\draw [brace_top] (3*1.5+0.03,-2.7) -- node [above, pos=0.5] {$ block_{n+1}$} 	(5*1.5-0.03,-2.7);

	    \draw  node[fill,circle,scale=0.6]at (3*1.5,-3.0)  {};%

		% low brace period
	
		\draw [brace_bottom] (1*1.5+0.03,-3.3) -- node [below, pos=0.5] {$e_{n}=d$} 	(3*1.5-0.03,-3.3);
		\draw [brace_bottom] (3*1.5+0.03,-3.3) -- node [below, pos=0.5] {$d$} 		(5*1.5-0.03,-3.3);

	\end{tikzpicture}
		\caption{Practical maximum playing error $(e)$}
	\end{figure}


        \ac{KMS} does not support playing video at higher speeds, a useful feature that enables users to watch a video in less time than its duration.
	For playing video with an higher velocity, we used \emph{ffmpeg}\footnote{\url{https://www.ffmpeg.org/}(accessed: 17 March 2016)} to convert the block into a new video with the desired velocity and seek time. Because the media duration is known, when the video started to being converted the headers located at the beginning of the file were already written and that made it possible to stream while converting.
	%{\color{red} [TALK ABOUT COMPLEXITY OF DECODE VS DECODE+MANIPULATE+ENCODE]}
	Although we implemented a solution that worked, we immediately noticed that \emph{ffmpeg} would take some time to initialize and that lead to pauses between switching parts.
        
	Later the \emph{Kurento} team released a version with support for seeking videos but we had to suspend the implementation of the fast forwarding feature as currently \ac{KMS} is not supporting that.
        %RP fiquei sem perceber. O método com o ffmpeg deixou de funcionar? Ou desististe devido ao outros problemas (delay, qualidade, cpu)?

	We could have implemented fast forwarding without real time conversion by creating multiple versions of the same video with different playback speeds after the recording of a block. When a user needed to play, he would also need to specify one of the available speed. We did not followed this approach has it would require larger disk space usage. 

\subsection{Server side recording to database}

	One of our concerns during the development of our solution was the storage scalability. Saving files directly into the file system would require an extra effort to distribute and replicate files among servers. For that purpose, the \emph{Kurento} team developed \emph{Kurento Repository}\footnote{\url{http://doc-kurento-repository.readthedocs.org} (accessed on 17 March 2016)} which is based on \emph{MongoDB}.

	One of the features that \emph{Kurento Repository} provides is the ability to play directly from the database without having to download the entire file to \ac{KMS}. The same is true for recording, but because the file headers are in the beginning and the file is written until is stops, the headers don't contain the necessary information for seeking the file.
	Although we gain with scalability with this approach, we lose access over the file for changing it to fit our needs, namely for using \emph{ffmpeg} or other video manipulation tool.
        
	Although we did not implement recorded file seeking, that could be achieved by waiting for full file recording and then proceed to database re-insertion with the correct headers. Another approach would be the specification of the file duration before recording, so the correct file headers could be written a priori. Both approaches were not possible to implement using just the \emph{Kurento} clients. We would need change the source code of \emph{Kurento} in order to add those new features.
        %RP explicar razões para não o fazer: perder upgrades, tornar instalação mais simples, maior complexidade manutenção app, etc.
        




