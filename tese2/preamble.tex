% ----------------------------------------------------------------------
% Define document language.
% ----------------------------------------------------------------------
% Select one
%\usepackage[latin1]{inputenc} % <<<<< Windows
\usepackage[utf8]{inputenc}   % <<<<< Linux

% Select one
%\usepackage[portuguese]{babel} % <<<<< Portuguese
\usepackage[english]{babel} % <<<<< English

% List of LaTeX variable names: \abstractname, \appendixname, \bibname,
%   \chaptername, \contentsname, \listfigurename, \listtablename, ...)
% http://www.tex.ac.uk/cgi-bin/texfaq2html?label=fixnam
%
% Changing the words babel uses (uncomment and redefine as necessary...)
\newcommand{\acknowledgments}{@undefined} % new LaTeX variable name
%
% > English
%
\addto\captionsenglish{\renewcommand{\acknowledgments}{Acknowledgments}}
%\addto\captionsenglish{\renewcommand{\listtablename}{List of Tables}}
%\addto\captionsenglish{\renewcommand{\listfigurename}{List of Figures}}
%\addto\captionsenglish{\renewcommand{\nomname}{Nomenclature}}
%\addto\captionsenglish{\renewcommand{\appendixname}{Appendix}}
%\addto\captionsenglish{\renewcommand{\bibname}{References}} % Bibliography

% > Portuguese
%
\addto\captionsportuguese{\renewcommand{\acknowledgments}{Agradecimentos}}
%\addto\captionsportuguese{\renewcommand{\listtablename}{Lista de Figuras}}
%\addto\captionsportuguese{\renewcommand{\listfigurename}{Lista de Tabelas}}
\addto\captionsportuguese{\renewcommand{\nomname}{Lista de S\'{i}mbolos}} % Nomenclatura
%\addto\captionsportuguese{\renewcommand{\appendixname}{Anexo}} % Apendice
%\addto\captionsportuguese{\renewcommand{\bibname}{Refer\^{e}ncias}} % Bibliografia


% ----------------------------------------------------------------------
% Define default and cover page fonts.
% ----------------------------------------------------------------------

% Use Arial font as default
%
\renewcommand{\rmdefault}{phv}
\renewcommand{\sfdefault}{phv}

% Define cover page fonts
%         encoding     family       series      shape
%  \usefont{T1}     {phv}=helvetica  {b}=bold    {n}=normal
%                   {ptm}=times      {m}=normal  {sl}=slanted
%                                                {it}=italic
% see more examples at
% http://julien.coron.free.fr/languages/latex/fonts/
%
\def\FontLn{% 16 pt normal
  \usefont{T1}{phv}{m}{n}\fontsize{16pt}{16pt}\selectfont}
\def\FontLb{% 16 pt bold
  \usefont{T1}{phv}{b}{n}\fontsize{16pt}{16pt}\selectfont}
\def\FontMn{% 14 pt normal
  \usefont{T1}{phv}{m}{n}\fontsize{14pt}{14pt}\selectfont}
\def\FontMb{% 14 pt bold
  \usefont{T1}{phv}{b}{n}\fontsize{14pt}{14pt}\selectfont}
\def\FontSn{% 12 pt normal
  \usefont{T1}{phv}{m}{n}\fontsize{12pt}{12pt}\selectfont}

% Define page margins and line spacing.
% set the page margins (2.5cm minimum in every side, as per IST rules)
\usepackage{geometry}
\geometry{verbose,tmargin=2.5cm,bmargin=2.5cm,lmargin=2.5cm,rmargin=2.5cm}

% Set space between lines.
% allow setting line spacing (line spacing of 1.5, as per IST rules)
\usepackage{setspace}
\renewcommand{\baselinestretch}{1.5}


% ----------------------------------------------------------------------
% Include external packages.
\usepackage{graphicx}

% Include acronyms
\usepackage{acronym}

% Mathematical enhancements for LaTeX.
\usepackage{amsmath}  % AMS mathematical facilities for LaTeX.
\usepackage{amsthm}   % Typesetting theorems (AMS style).
\usepackage{amsfonts} %



% 'subfigure' package
%
% Deprecated: Figures divided into subfigures.
% http://www.ctan.org/tex-archive/obsolete/macros/latex/contrib/subfigure/
%
% > subcaptions for subfigures
%
\usepackage{subfigure}


% 'subfigmat' package
%
% Automates layout when using the subfigure package.
% http://www.ctan.org/tex-archive/macros/latex/contrib/subfigmat/
%
% > matrices of similar subfigures
%
\usepackage{subfigmat}


% 'url' package
%
% Verbatim with URL-sensitive line breaks.
% http://www.ctan.org/tex-archive/macros/latex/contrib/url/
%
% > URLs in BibTex
%
% \usepackage{url}


% 'rotating' package
%
% Rotation tools, including rotated full-page floats.
% http://www.ctan.org/tex-archive/macros/latex/contrib/rotating/
%
% > show wide figures and tables in landscape format:
%   use \begin{sidewaystable} and \begin{sidewaysfigure}
%   instead of 'table' and 'figure', respectively.
%
\usepackage{rotating}


% 'hyperref' package
%
% Extensive support for hypertext in LaTeX.
% http://www.ctan.org/tex-archive/macros/latex/contrib/hyperref/
%
% > Extends the functionality of all the LATEX cross-referencing
%   commands (including the table of contents, bibliographies etc) to
%   produce \special commands which a driver can turn into hypertext
%   links; Also provides new commands to allow the user to write adhoc
%   hypertext links, including those to external documents and URLs.
%
\usepackage[pdftex]{hyperref}              % enhance documents that are to be output as HTML and PDF
\hypersetup{colorlinks,                    % color text of links and anchors, eliminates borders around links
            % linkcolor=red,               % color for normal internal links
            linkcolor=black,               % color for normal internal links
            anchorcolor=black,             % color for anchor text
            % citecolor=green,             % color for bibliographical citations
            citecolor=black,               % color for bibliographical citations
            % filecolor=magenta,           % color for URLs which open local files
            filecolor=black,               % color for URLs which open local files
            % menucolor=red,               % color for Acrobat menu items
            menucolor=black,               % color for Acrobat menu items
            % urlcolor=cyan,               % color for linked URLs
            urlcolor=black,                % color for linked URLs
            % bookmarks=true,              % create PDF bookmarks (by default is true)
	        bookmarksopen=false,           % don't expand bookmarks
	        bookmarksnumbered=true,        % number bookmarks
	        pdftitle={Thesis},             % TODO Change to your thesis title
            pdfauthor={TODO},              % TODO Change to your name
            pdfsubject={TODO },            % TODO Change to your thesis subject
            pdfkeywords={TODO },           % TODO Change to thesis keywords
            pdfstartview=FitV,
            pdfdisplaydoctitle=true}


% 'hypcap' package
%
% Adjusting the anchors of captions.
% http://www.ctan.org/tex-archive/macros/latex/contrib/oberdiek/
%
% > fixes the problem with hyperref, that links to floats points
%   below the caption and not at the beginning of the float.
%
\usepackage[figure,table]{hypcap}

% ----------------------------------------------------------------------
% Define new commands to assure consistent treatment throughout document
% ----------------------------------------------------------------------
\newcommand{\degree}{\ensuremath{^\circ\,}} % degrees


% If you need a numbered subsubsubsection uncomment the following lines
% \setcounter{secnumdepth}{4}
% \setcounter{tocdepth}{2}                 % Refrains ToC to two levels
% \def\subsubsubsection#1{\paragraph{#1}~\\}



\usepackage{amssymb}
\usepackage{makeidx}  % allows for indexgeneration
\usepackage{kmath}  % allows for indexgeneration
\usepackage{color}  % allows for indexgeneration
\usepackage[dvipsnames]{xcolor}
\usepackage{url}
\usepackage{array}
\usepackage{hyperref}
\usepackage{graphicx}
\usepackage{caption}

\usepackage{float}
\usepackage{mathtools}
\usepackage{acronym}
\usepackage{pgfgantt}
\usepackage{adjustbox}
\usepackage{rotating}
\usepackage{listings}
\usepackage{dirtytalk}
\usepackage[graphicx]{realboxes}
\definecolor{fade}{rgb}{0.8,0.8,0.8}
\usepackage{tabularx} % in the preamble
\usepackage{xcolor}

\usepackage{tikz}
\usepackage{pgf-umlsd}
\usepackage{textcomp}
\usepackage{environ}
\usepackage{graphics}
\usepackage{pgfplots} 
\pgfplotsset{compat=1.9}

\newcolumntype{M}[1]{>{\centering\arraybackslash}m{\linewidth/#1}}


%\overfullrule=2cm

\usetikzlibrary{positioning,shapes,shadows,arrows,matrix,decorations.pathreplacing}


\newganttchartelement*{mymilestone}{
mymilestone/.style={
shape=isosceles triangle,
inner sep=0pt,
draw=cyan,
top color=white,
bottom color=cyan!50
},
mymilestone incomplete/.style={
/pgfgantt/mymilestone,
draw=yellow,
bottom color=yellow!50
},
mymilestone label font=\slshape,
mymilestone left shift=0pt,
mymilestone right shift=0pt
}

\newgantttimeslotformat{stardate}{%
\def\decomposestardate##1.##2\relax{%
\def\stardateyear{##1}\def\stardateday{##2}%
}%
\decomposestardate#1\relax%
\pgfcalendardatetojulian{\stardateyear-01-01}{#2}%
\advance#2 by-1\relax%
\advance#2 by\stardateday\relax%
}




\lstdefinelanguage{JavaScript}{
  keywords={break, case, catch, continue, debugger, default, delete, do, else, false, finally, for, function, if, in, instanceof, new, null, return, switch, this, throw, true, try, typeof, var, void, while, with},
    numbers=left,
    numberstyle=\scriptsize,
  morecomment=[l]{//},
  morecomment=[s]{/*}{*/},
  morestring=[b]',
  morestring=[b]",
      stepnumber=1,
    numbersep=8pt,
    showstringspaces=false,
    breaklines=true,
    frame=lines,
    backgroundcolor=\color{background},
  ndkeywords={class, export, boolean, throw, implements, import, this},
  keywordstyle=\color{blue}\bfseries,
  ndkeywordstyle=\color{darkgray}\bfseries,
  identifierstyle=\color{black},
  commentstyle=\color{purple}\ttfamily,
  stringstyle=\color{red}\ttfamily,
  sensitive=true
}

\usepackage{color, colortbl}
\definecolor{Gray}{gray}{0.9}

\colorlet{punct}{red!60!black}
\definecolor{background}{HTML}{FFFFFF}
\definecolor{delim}{RGB}{20,105,176}
\colorlet{numb}{magenta!60!black}
\definecolor{grisbg}{gray}{0.95}

\lstdefinelanguage{json}{
    %basicstyle=\normalfont\ttfamily,
  keywords={data, cmd},
    numbers=left,
    numberstyle=\scriptsize,
    morestring=[b]',
    morestring=[b]",
    stepnumber=1,
    numbersep=8pt,
    showstringspaces=false,
    breaklines=true,
    frame=lines,
    backgroundcolor=\color{background},
    literate=
     *{0}{{{\color{numb}0}}}{1}
      {1}{{{\color{numb}1}}}{1}
      {2}{{{\color{numb}2}}}{1}
      {3}{{{\color{numb}3}}}{1}
      {4}{{{\color{numb}4}}}{1}
      {5}{{{\color{numb}5}}}{1}
      {6}{{{\color{numb}6}}}{1}
      {7}{{{\color{numb}7}}}{1}
      {8}{{{\color{numb}8}}}{1}
      {9}{{{\color{numb}9}}}{1}
      {:}{{{\color{punct}{:}}}}{1}
      {,}{{{\color{punct}{,}}}}{1}
      {\{}{{{\color{delim}{\{}}}}{1}
      {\}}{{{\color{delim}{\}}}}}{1}
      {[}{{{\color{delim}{[}}}}{1}
      {]}{{{\color{delim}{]}}}}{1},
 keywordstyle=\color{blue}\bfseries,
  ndkeywordstyle=\color{darkgray}\bfseries,
  identifierstyle=\color{black},
  commentstyle=\color{purple}\ttfamily,
  stringstyle=\color{red}\ttfamily,
  sensitive=true
}


