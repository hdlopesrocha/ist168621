\section{Solution deployment}
	\label{section:deployment}

	In this section we describe the hardware and software we have used to deploy our solution. 

	\subsection{Hardware}

	The hardware was gently provided by INESC-ID\footnote{\url{http://www.inesc-id.pt/} (Accessed March 26, 2016)}, the specifications of the server that we used are specified on table \ref{table:hwspecs}.

	\begin{table}[H]
\centering
\caption{Hardware specifications}
\label{table:hwspecs}
\begin{tabular}{|l|l|}
\hline
Server & Supermicro SuperServer 6027R-72RF\footnotemark \\ \hline
CPU & 2 x Intel Xeon E5-2640V2, LGA 2011, 2.0GHz, 8C/16T \\ \hline
RAM & 8 x DDR3 REG16G-1600DDR3, 16GB, DDR3-1600, Registered ECC, memory \\ \hline
Network cards & 2 x Intel Corporation I350 Gigabit Network Connection (rev 01)  \\ \hline
Disks & \begin{tabular}[c]{@{}l@{}}2x SAMSUNG SSD 840 PRO 256GB SATA III (Drive 0 - RAID 1 - 237.486 GB)\\
4x WESTERN DIGITAL 3TB SATA III 64MB RED (Drive 1 - RAID 5 - 8.185TB)\end{tabular} \\ \hline
\end{tabular}
\end{table}

 \footnotetext{\url{http://www.supermicro.com/products/system/2U/6027/SYS-6027R-72RF.cfm} (Acessed March 26, 2016)}

\subsection{Operating System}
	
On table \ref{table:osspecs} we present an overview of the operating system configurations of the machine we used for deployment.

	\begin{table}[H]
\centering
\caption{Operating system specifications}
\label{table:osspecs}
\begin{tabular}{|l|l|}
\hline
Operating System & Linux version 3.16.0-4-amd64 \\ \hline
Distribution & Debian GNU/Linux 8 (jessie)\footnotemark \\ \hline
Swap & 16GB (Drive 0)\\ \hline
Root & 40GB (Drive0)\\ \hline
EFI & 200MB (Drive 0)\\ \hline
Bcache & Remaining space (Drive 0 \& Drive 1) \\ \hline
\end{tabular}
\end{table}

\footnotetext{\url{https://www.debian.org/} (Accessed March 26, 2016)}

\subsection{Software}
	INESC-ID provided us a restricted linux account without administration permissions, which we prevented us the installation of our solution directly on the machine because we would need administrative priviledges to install all the software that our solution depends on. Although that limitation, they provided access to Docker\footnote{\url{https://www.docker.com/} (Accessed March 27, 2016)} which run on host operating system as an isolated process in userspace.

	Inside a Docker container we have the administrative priviledges to install all the software dependencies.

	We could use one docker image for each component but in order to reduce the network usage we prefered to install all the components within the same image. At the same we provide an all-in-one easy to deploy solution as all the \ac{IP}s are local.

	We present, on table \ref{table:softspecs}, the software we have installed inside our docker container including the versions that are in use.

	\begin{table}[H]
\centering
\caption{Installed Software}
\label{table:softspecs}
\begin{tabular}{|l|l|}
\hline
\multicolumn{1}{|c|}{\textbf{Name}} & \multicolumn{1}{c|}{\textbf{Version}}         \\ \hline
Ubuntu Server\footnotemark & 14.04 LTS   \\ \hline
Oracle Java & 1.8.0\_77   \\ \hline
MongoDB & 3.0.10   \\ \hline
Kurento Media Server & 6.4.0 \\ \hline
Kurento Repository & 6.3.1 \\ \hline
Python & 2.7.6 \\ \hline
\end{tabular}
\end{table}
 \footnotetext{\url{http://www.ubuntu.com/} (Accessed March 27, 2016)}



