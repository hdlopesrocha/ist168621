%RP antes desta secção fala um introdução ao capítulo a explicar o que vai ser apresentado.
%HR done

\section{Data Model}
The data model is a critical component of our solution, as a badly designed model can imply serious difficulties when implementing new features that are not part of the plans. During the course of this project, we had to redesign the model more than once in order to support new features.

%RP devias começar com um overview dos dados que é necessário guardar para a aplicação
%RP ao apresentar o modelo falas de coisas como mensagens para grupos, mas ainda não sabemos nada disso. Na arquitectura, devias falar da concepção, do modelo de comunicação. Que tens utilizadores, grupos, mensagens, chats, convites, etc...
%HR não meti na arquitectura mas adicionei a linha seguinte

In order to offer all the functionalities that we promise, some information about objects must be persistent such as users, groups,relations among users, group memberships, messages, hyper-content, recordings and collaborative editor state.

\subsection {Schema representation}
    \emph{MongoDB} has a slightly different terminology from relational databases. The first big difference is instead of having tables \emph{MongoDB} stores its objects on collections. The analogous data structure to the table row is a document.

    Each \emph{MongoDB} document is represented by a \emph{JSON} object and, as a result, each document may have different attributes within a collection. Needless to say that, by following this approach we don't need to create the collections with a well predefined schema. In fact we don't need to define it at all but, for reasons of coherence and organization, we represent our database collections as they would have a predefined schema by following the same document structure as we will present in this section.
    %RP a última frase é confusa. O que queres dizer? Reescreve
    %HR está melhor?

    Similarly to relational databases, \emph{MongoDB} requires a primary key (typically of type \emph{ObjectId} and named \emph{"\_id"}) for each document, which is automatically assigned if not specified. 

    In order to define \emph{foreign keys}, we just store them as \emph{ObjectId}s if and only if the foreign keys point to documents within a unique collection, otherwise we need an additional attribute to specify which collection is the \emph{foreign key} pointing at.

    In respect to attributes nullability, \emph{MongoDB} does not enforce a document's attribute to have a not null value, although, for sake of good functionality, we perform those constraints validation programmatically and as such we also represent them in our schema.
    
    An example of schema representation can be seen on table \ref{table:schema}.

\begin{table}
\centering
\caption{Schema representation}
\label{table:schema}
    \begin{tabular}{|ll|}
        \hline
        \multicolumn{2}{|c|}{\textbf{Collection name}}            \\ \hline
        $\Diamondblack$ \underline{\_id (primary key)} & ObjectId \\ 
        $\medbullet$ Not nullable property name & Property type   \\ 
        $\medcirc$ Nullable property name       & Property type   \\
        $\medcirc$ Reference to document        & ObjectId        \\
        $\medbullet$ Embedded document          & Document        \\ 
        $\medbullet$ Embedded list              & List[Type]      \\ \hline
    \end{tabular}
\end{table}

\subsection {Generic model}

For designing our model we have taken into account generic programming techniques. We observed that operations like searching for an object were quite repeated across different types of objects. 

Our first decision for our model, in order to avoid repeated code, was the isolation of the object's attributes from themselves, so we could apply the search operation to a set of attributes independently of the object type. To this generic set of properties we call data (Table \ref{table:generic}) and each object of this type has a reference to the owner, which is a unique identification number.

\begin{table}
\centering
\caption{Generic data model}
\label{table:generic}
    \begin{tabular}{cc}
        \begin{tabular}{|ll|}
            \hline
            \multicolumn{2}{|c|}{\textbf{Data}}          \\ \hline
            $\Diamondblack$ \underline{\_id} & ObjectId  \\ 
            $\medbullet$ owner          & ObjectId       \\ 
            $\medbullet$ properties    & List[Attribute] \\ 
            $\medcirc$ searchableValues & List[Text]     \\ \hline
        \end{tabular}
        \begin{tabular}{|ll|}
            \hline
            \multicolumn{2}{|c|}{\textbf{Attribute}}     \\ \hline
            $\medbullet$ key            & Text           \\ 
            $\medcirc$ value            & Object         \\ 
            $\medbullet$ identifiable   & Boolean        \\ 
            $\medcirc$ readPermissions  & List[ObjectId] \\ 
            $\medcirc$ writePermissions & List[ObjectId] \\ \hline
        \end{tabular}    
    \end{tabular}
\end{table}

The owner's identification number by itself is not sufficient to identify an object, as objects from different types can have the same identification number. In order to solve this problem, when an object is created, its correspondent data must contain the owner's object type. 
%RP mas o _id é apresentado com sendo uma chave primária na figura 4.2, logo única!
%HR "The owner's identification number" corriji, não me referia ao "_id"

Whenever an attribute is created, its name, value and indentifiability must be specified. The set of objects that can read and write that attribute must also be defined. 
In regard to our permission mechanism, if the read or write sets are not specified we assume the attribute is readable and writable by everyone. Conversely if the read and write sets are empty, nobody is allowed to read or write the attribute. Implicitly, if an entity can write an attribute, it can also read it.

In particular, if all attributes were searchable, it could be simple to search for attributes that could reveal sensible information about an object. For example if we consider that a user could have a health related attribute, searching by a disease would reveal which users could suffer from a certain disease. The leak of that kind of information could, for instance, change the agreement between users and health insurance companies. For this reason only the specified attributes as searchable will be taken into account when performing keyword searches.
%RP era melhor que o exemplo fosse relativo à tua aplicação e não algo abstracto como saúde.
%RP Dás permissões de leitura e escrita a objectos. Normalmente é a utilizadores. Explicar relação. Assumo que sejam aqui referenciados apenas objectos que representam utilizadores

Another important attribute specification is the owner identifiability, which tells us if the attribute identifies the object. This specification lets us create abstract authentication services. For example a user can login into our system by providing any attribute that identifies himself, e.g. the e-mail but others are possible like the user name or cellphone number. 

Not less important, in order to get an object's properties efficiently, we have created an index over the \emph{owner} attribute. We have also created an index over the \emph{searchableValues} in order to improve the keyword search performance.

In summary, with this model we can perform search and identification of any kind of objects, as we will see on the following models. In particular, the \emph{user} and \emph{group} models are using this generic model for storing their attributes. 
% Not less important, attributes can specify aggregations of objects. For example, the user role is an aggregator property that within users it allows the identifaction of each one is administrator. This aggregation specification is independent from the object type, so it's possible to search for the administrative role and return users and groups of users that contains that property. <= [currently this operation is possible but it is not specified]


%RP antes de falares de cada modelo, que tal incluir uma figura com a relação entre os vários tipos de dados. Mesmo não sendo SQL, as relações existem conceptualmente e falas em chaves estrangeiras, pelo que tal figura faz sentido e ajuda a perceber.

\subsection{User model}

The user model is not tied to the user attributes (Table \ref{table:user}), the information maintained in this model is just used for authentication purposes. Passwords are not stored in plain text, instead we apply hashing and salting techniques \cite{password} in order to make it harder to decode the password by an attacker. Accordingly, we use \emph{SHA-1} and a random salt per user with 32 characters.
%RP Table x, Figure y, etc é sempre maiusculas.

\begin{table}
\centering
\caption{User model}
\label{table:user}
    \begin{tabular}{|ll|}
        \hline
        \multicolumn{2}{|c|}{\textbf{User}}         \\ \hline
        $\Diamondblack$ \underline{\_id}  & ObjectId  \\ 
        $\medbullet$ hash           & Text          \\ 
        $\medbullet$ salt      & Text               \\ \hline
    \end{tabular}
\end{table}

\subsection{Relation model}

A relation between two entities $e_1$ and $e_2$ is represented by the pair $e_1\rightarrow e_2$ (Table \ref{table:relation}), where $e_1$ is the source and $e_2$ is the target. This relation is said to be bi-directional if and only if it also exists the relation $e_2\rightarrow e_1$.

\begin{table}
\centering
\caption{Relation model}
\label{table:relation}
    \begin{tabular}{|ll|}
        \hline
        \multicolumn{2}{|c|}{\textbf{Relation}}     \\ \hline
        $\Diamondblack$ \underline{\_id}  & ObjectId  \\ 
        $\medbullet$ source           & ObjectId    \\ 
        $\medbullet$ target      & ObjectId         \\ \hline
    \end{tabular}
\end{table}

A user can only interact with friends or with group members. In order to validate a friendship, both users must agree on that friendship, in other words it, there must exist a bi-directional relation between both users.

In order to improve the performance of queries over the \emph{Relation} collection we have created indexes on the \emph{source}, \emph{target} and also on the pair composed by both attributes.

\subsection{Group model}

A conference room, which is the environment where users communicate among their selves, is represented persistently by a group of participating entities (in general users but our system allow other types of entities).

A group is composed by an \emph{id}, \emph{inviteToken} and a \emph{visibility} as shown on table \ref{table:group}.
Moreover, a group can be public or private. If the group is public, then it is visible to all users that maintain a friendship with a member of this group. If the group is private, then it is only visible to its members.

\begin{table}
\centering
\caption{Group model}
\label{table:group}
    \begin{tabular}{cc}
        \begin{tabular}{|ll|}
            \hline
            \multicolumn{2}{|c|}{\textbf{Group}}        \\ \hline
            $\Diamondblack$ \underline{\_id}  & ObjectId  \\ 
            $\medcirc$ inviteToken           & Text     \\ 
            $\medbullet$ visibility      & Text         \\ \hline
        \end{tabular}

        \begin{tabular}{|ll|}
            \hline
            \multicolumn{2}{|c|}{\textbf{GroupMembership}}  \\ \hline
            $\Diamondblack$ \underline{\_id}  & ObjectId      \\ 
            $\medbullet$ groupId            & ObjectId      \\ 
            $\medbullet$ userId & ObjectId                  \\ \hline
        \end{tabular}    
    \end{tabular}
\end{table}


The group membership is a special case of relation, where the target entity is always a group.
When a group is created, a group membership is automatically assigned to its creator.

Entities that have a membership with a group can create more memberships by sharing an invite token or by specifying new group members. This invite token is used to create an invite \ac{URL} that if shared with other users allow them to join the group. Invite tokens can be deleted or regenerated with a different value.
%RP isto quer dizer que quem tem o token se pode juntar ao grupo? Explicar. O que impede alguém de o partihar com outros?
%HR done

In order to improve performance of queries over \emph{GroupMembership} collection we have created indexes on the \emph{groupId}, \emph{userId} and also on the pair composed by both attributes.


\subsection{Message model}

A message is composed by its content, time of creation and source and target identification numbers (Table \ref{table:message}). The message's target could reference any object, but our application is only handling messages to groups.

\begin{table}
\centering
\caption{Message model}
\label{table:message}
    \begin{tabular}{|ll|}
        \hline
        \multicolumn{2}{|c|}{\textbf{Message}}      \\ \hline
        $\Diamondblack$ \underline{\_id}  & ObjectId  \\ 
        $\medbullet$ source           & ObjectId    \\ 
        $\medbullet$ target      & ObjectId         \\ 
        $\medbullet$ content      & Text            \\ \hline
    \end{tabular}
\end{table}

In order to query for recent messages for a given target, we use the oldest message's \emph{\_id}, which is sequential, in order to find messages with a newer \emph{\_id}.

In order to improve performance of queries over \emph{Relation} collection we have created indexes on the \emph{target} attribute and also on the pair composed by \emph{target} and \emph{\_id} attributes for finding and sorting the messages received by an entity more efficiently. 

%RP pela justificação não consigo perceber para que serve a chave no par target e _id
%HR in order to find recent messages (não tinha explicado)

\subsection{Hyper content model}

During a group conversation, it is possible to create time annotations for making it easy to access that time either for searching or sharing with other users.
A time annotation (table \ref{table:hyper}) contains a title, the correspondent group identification number and the time itself.

The hyper content is used to synchronize content among users during a conversation. Table \ref{table:hyper} shows that every hyper content must have a start and ending time, the correspondent group identification number and the content itself, in the form of text. Beside those attributes, in order to perform queries over the \ac{HTML} contents with more precision, we have added an additional \emph{searchableContent} attribute. It contains just the searchable content extracted from the content, by excluding the \ac{HTML} tags and parameters using \emph{Jsoup}\footnote{\url{http://jsoup.org/} (Accessed April 11, 2016)}.

In the point o view of a user, a time annotation is just a simple way to associate a time stamp to a topic in order to make it easier to find. On the other hand, the hyper-content is more complete than time annotations, except they are not visible on the time-line. In addition to time annotations, hyper-contents have a duration and content that can be superimposed to video.

%RP explica melhor a diferença entre HyperContent e Time Annotation do ponto de vista do utilizador / funcionalidade proporcionada.
%HR done

\begin{table}
\centering
\caption{Hyper content model}
\label{table:hyper}
    \begin{tabular}{cc}
        \begin{tabular}{|ll|}
            \hline
            \multicolumn{2}{|c|}{\textbf{HyperContent}}  \\ \hline
            $\Diamondblack$ \underline{\_id}  & ObjectId \\ 
            $\medbullet$ groupId            & ObjectId   \\ 
            $\medbullet$ start              & Date       \\ 
            $\medbullet$ end                & Date       \\ 
            $\medbullet$ content            & Text       \\ 
            $\medbullet$ searchableContent  & Text       \\ \hline
        \end{tabular}

        \begin{tabular}{|ll|}
            \hline
            \multicolumn{2}{|c|}{\textbf{TimeAnnotation}}  \\ \hline
            $\Diamondblack$ \underline{\_id}  & ObjectId   \\ 
            $\medbullet$ groupId            & ObjectId     \\ 
            $\medbullet$ title & Text                      \\ 
            $\medbullet$ time & Date                       \\ \hline

        \end{tabular}    
    \end{tabular}
\end{table}

For example the content showed in listing \ref{lst:jsoup} would produce the searchable content "Click here!".

\begin{minipage}{\linewidth}
\begin{lstlisting}[caption={Example of HTML content},label={lst:jsoup},language=JavaScript]
<div class="subtitle">
    <a href="/">Click here!</a>
</div>
\end{lstlisting}
\end{minipage}

In order to improve the performance of searching content and calculate interval intersections, we have created indexes over the following sequences of attributes: 

\begin{itemize}
\item{\emph{\{groupId,start,end\}} $\rightarrow$ for finding visible contents for a given time  within a group (intersections).}
\item{\emph{\{groupId,start\}} $\rightarrow$  for finding contents that starts after a given time within a group (for pre-loading).}
\item{\emph{\{groupId,searchableContent\}} $\rightarrow$ for searching contents by keywords within a group.}  
\end{itemize}

\subsection{Collaborative Content model}

Within a conversation, users can write documents collaboratively. As we can see in Table \ref{table:collaborative}, each document has a content and a reference to the correspondent group. 


\begin{table}
\centering
    \caption{Collaborative content model}
    \label{table:collaborative}
    \begin{tabular}{|ll|}
        \hline
        \multicolumn{2}{|c|}{\textbf{CollaborativeContent}} \\ \hline
        $\Diamondblack$ \underline{\_id}  & ObjectId        \\ 
        $\medbullet$ groupId            & ObjectId          \\ 
        $\medcirc$ content            & Text                \\ \hline
    \end{tabular}
\end{table}

In order to improve the performance of finding the group's content we have created and index over the \emph{groupId} attribute.

\subsection{Recording model}

During a conversation, users may allow sharing their web cameras. By doing so, their video is stored in recording chunks. Each chunk, described in Table \ref{table:recording}, represents an interval of time $T=\big[c^{start},c^{end}\big[$ of the video stream. It also contains the media's \ac{URL}, a reference to a group, an owner identification number and the correspondent \emph{WebSocket} session id.
    
    The media's \ac{URL} is the location where \emph{Kurento Repository} stores one chunk of video (including audio) which is used to playback on users demand.

    %RP ainda não tenho forma de perceber o que é o websocket session id. Nunca falaste de websockets antes, quanto mais disto! Nem do media URL, é gravado ou webrtc (dispositivo do utilizador)?
    
    In order to allow the same user to have different devices, storing just the \ac{URL} with an associated user id is not enough. In this case, one more parameter is needed to differentiate the different \emph{webSocket} sessions opened by the same user. For this reason we had to associate a random session identification number to each \emph{webSocket}.

\begin{table}
\centering
    \caption{Recording model}
    \label{table:recording}
    \begin{tabular}{cc}
        \begin{tabular}{|ll|}
            \hline
            \multicolumn{2}{|c|}{\textbf{RecordingInterval}}  \\ \hline
            $\Diamondblack$ \underline{\_id}  & ObjectId      \\ 
            $\medbullet$ groupId            & ObjectId        \\ 
             $\medbullet$ start             & Date            \\ 
            $\medbullet$ end                & Date            \\ \hline
        \end{tabular}   
        \begin{tabular}{|ll|}
            \hline
            \multicolumn{2}{|c|}{\textbf{RecordingChunk}}  \\ \hline
            $\Diamondblack$ \underline{\_id}  & ObjectId     \\ 
            $\medbullet$ groupId            & ObjectId     \\ 
            $\medbullet$ owner           & ObjectId        \\ 
            $\medbullet$ sessionId          & Text         \\
            $\medbullet$ start              & Date         \\ 
            $\medbullet$ end                & Date         \\ 
            $\medbullet$ url               & Text          \\ \hline
        \end{tabular}
    \end{tabular}
\end{table}



A set of chunks $S=[c_1,c_2,\ldots,c_n]$ is said continuous if $\forall c_i\in S, \exists c_j \in S$ where $j\neq i$ and $c_i^{start} = c_j^{end} \vee c_i^{end} = c_j^{start}$. 
A recording interval represents a continuous set of recording chunks.

In respect to find which chunks to play given a specified time (which is the current time in a user's perspective), we must select the chunk with an ending instant immediately after the current time, if the chunk's beginning is placed before the current time we have to play it instantaneously otherwise the chunk's beginning will occur after the current time and therefore we have to schedule the playback.

On the other hand, we use the fact that \emph{\_id}'s are sequential in order to determine adjacent chunks, for example, if we need to find the next chunk to play, we have to sort all elements by \emph{\_id} (which on \emph{MlongoDB} is already sorted by default) and select a chunk with an \emph{\_id} immediately after the current one. 

In order to improve the performance of searching recording chunks and calculate which chunks are going to be reproduced, we have created indexes over the following sequences of attributes: 
\begin{itemize}
    \item{\emph{\{groupId,start,end\}} $\rightarrow$ for finding all available chunks for a given time} 
    \item{\emph{\{groupId,sessionId,end\}} $\rightarrow$ for finding chunks that ends after a given time within a session}
    \item{\emph{\{groupId,owner,end\}} $\rightarrow$ for finding chunks that ends after a given time within a user (for a different session) or group}
    \item{\emph{\{groupId,sessionId,\_id\}} $\rightarrow$ for finding chunks that follow a given \emph{id} within a session}
    \item{\emph{\{groupId,owner,\_id\}} $\rightarrow$ for finding chunks that follow a given \emph{id} within a user or group}
\end{itemize}
%RP os _id não são únicos? Para que servem os index que os incluem?
%HR done, for finding adjacent chunks

%RP para a intercepção não precisaste de usar o start, apenas o end?
%HR exactly, if there is no intersection taking into account the "end" at least I can schedulle the next player (after the end)


Not less important, we have also created an index over the \emph{RecordingInterval}'s \emph{groupId} attribute for listing all intervals within a conference room. 
%RP olha outra coisa nova! Ainda não falaste de conference rooms!
%HR "conference room" dá-se no contexto de um "group", era para não estar sempre a repetir. Qualquer das formas adicionei esse pormenor na seccao do grupo.


